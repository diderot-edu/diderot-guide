\chapter{Diderot Exams and Quizzes}
\label{ch:quiz}
 
\begin{preamble}
You can use Diderot to administer tests, quizzes, and exams.  This chapter presents a brief these work on Diderot.
\end{preamble}


\section{Submittable Chapters}

The basic idea is to create a \defn{submittable chapter} on Diderot and upload the test problems onto the chapter.  To create the test problems, the author can use LaTeX.  The supported problem types include
\begin{itemize}
\item multiple choice (including true/false) problems, where the test taker picks one of the
  choices
\item multiple-answer problems, where the test-taker can choose more than one of the choices
\item free response problems, where the test taker writes their answer in a designated area
\item fill-in-the-blanks problems, where the test taker fills in the missing parts of a solution.
\end{itemize}


Diderot supports using mathematics (using MathJax) and attaching images into answer.  Algorithms or code in fixed-font can be included in all problems, including in fill-in-the-blanks question.

\section{Type Setting Quizzes Using LaTeX}


The author can attach to any atom a sequence of \defn{prompts}, which come in two flavors: questions prompts and answer prompts.
%

\defn{Question prompts}  include 
%
\lstinline`\ask` for free-response questions, 
%
\lstinline`\onechoice` for multiple choice questions with a single correct answer, 
%
\lstinline`\anycoice` for multiple answer questions with any number correct answers, and 
%
\lstinline`\asktf` for true-false questions.

\defn{Solution prompts}  include 
%
\lstinline`\sol` for free-response solution, 
%
\lstinline`\solfin` for fill-in-the-blanks solution, 
%
\lstinline`\choice` and \lstinline`\choice*` for an incorrect and correct choice, 
%
\lstinline`\solt` and \lstinline`\solf`   for true and false answers.

A question typically consisting of a question prompt followed by any
number of solution prompts.  
%
Any number of questions may be attached  to an atom.

There are two ways to specify point values.
%

\begin{gram}[Point Values]
The author can assign integer point values to each atom. The point value must be specified with a period at the end should be the first argument to the atom.
%
For example \lstinline`\begin{problem}[10.]` specifies a  problem with ten points, and
%
\lstinline`\begin{problem}[10.][Problem Title]` specifies a  problem with ten points and and the title
\lstinline`Problem Title`.
%
By default an atom's points are distributed to its questions evenly,
which can result in non-integral, even ``irrational'' point scores, such as $3.33333333...\ldots 3$. 
%
\end{gram}


\begin{gram}[Question Factors]
The author can specify a real-valued \defn{factor} for each question prompt.  For
example, For example \lstinline`\ask[2.0]` specifies a factor of $2.0$
for this question prompt. If there are multiple questions in an atom,
their factor values are added and the point value of the atom is
distributed over the questions in proportion with their factors.  For
example, if a problem atom is worth $9$ points and its two questions
prompts have factors $2.0$ and $1.0$, then the first question will be
assigned $6.0$ points and the second question will be assigned $3.0$
points.

If a question does not have a factor, it is given a default factor of $1.0$. 
\end{gram}

\begin{gram}[Solution  Factors]
The author can specify a real-valued \defn{factor} for each solution prompt.  For
example, For example \lstinline`\sol[0.5]` specifies a factor of $0.5$
for this solution prompt. If there are multiple solution prompts
belonging to a question prompt, then their factors are added and the
factor of question is distributed over the solutions in proportion
with their factors.  For example, if a question has a factor of $9.0$
and its two solutions prompts have factors $0.5$ and $1.0$, then
the first solution prompt will be assigned $3.0$ factors and the second question
prompt will be assigned $6.0$ factors.  The point value  for each solution prompt will in turn be determined by the point value of the atom.

By default free-response solution have a default factor of $1.0$ and
so are correct-choice prompts.  Incorrect choice prompts have a factor value of $0.0$
\end{gram}


%
\lstinline`\begin{problem}[10.][Problem Title]` specifies a  problem with ten points and and the title
\lstinline`Problem Title`.

\begin{example}[Free response with a single solution]
The code below will generate a question and a text box for the student
to write their answer.

\begin{lstlisting}
\begin{problem}
Some flavor text...

\ask Now comes the questions.  Please write your answer into the
provided space below.

\sol This is the solution. Diderot will leave space proportional to
the length of the solution for the student to answer.  If you don't
have the solution, simply use \~ to indicate the length.  For example,
like this:~~~~~~~~~~~~~~~~~~~~~~~~~~~~~~
\end{problem}
\end{lstlisting}

\end{example}

\begin{example}[Free response with a multiple solutions]

A question can have multiple solution prompts.  The example below has
two free-response solution prompts. For this question, Diderot will
display two text boxes for the students to enter their answers.
%
Because this problem does not specify a point value, the factor $4.0$
of the question prompt becomes its point value.
%
Because there are two solution boxes, each is worth $2.0$ points.

\begin{lstlisting}
\begin{problem}
Some flavor text...

\ask[4.] Now comes the question.  This question requires the student to
answer two questions.  Please write your answers to each part in the
provided space below.

\sol This is the solution for the first part. Diderot will leave space
proportional to the length of the solution for the student to answer.
If you don't have the solution, simply use ~ to indicate the length.
For example, like this:~~~~~~~~~~~~~~~~~~~~~~~~~~~~~~

\sol This is the solution for the second part. Diderot will leave
space proportional to the length of the solution for the student to
answer.  If you don't have the solution, simply use ~ to indicate the
length.  For example, like this:~~~~~~~~~~~~~~~~~~~~~~~~~~~~~~

\end{problem}
\end{lstlisting}

\end{example}

\begin{example}[Fill-in-the-Blanks questions]
Sometimes it is easier to formulate a question as a fill-in-the-blanks
question.  To this, the author can use \lstinline`\solfin`
(solution-fill-in) prompt, and
%
use the command \lstinline`\fin` (short for fill-in) to indicate the
parts that must be blanked  out for  the students to fill in.
%
The following example requires the student
to fill in three blanks for a total of $6$ points.

\begin{lstlisting}
\begin{problem}[6.]
Consider the following statement: all prime numbers are odd.

\ask Please indicate if the statement above is correct or incorrect
and justify your answer.

\solfin To be prime a number, the number must have \fin{no divisors
  other than itself and one}.

The above statement is \fin{incorrect}, because \fin{if the
  number is even, it can divide itself and this does not count against
  it}.

If you have answered incorrect, you can also provide a counterexample
here: \fin{2~~~~~~}.  (We are adding some additional space to the
solution so that we don't give any tips about the solution via the
length of the space).
\end{problem}
\end{lstlisting}
\end{example}


\begin{example}[Multi-Part Fill-in-the-Blanks Questions]
You can ask multiple-part fill in the blanks questions as follows.
The following example has 4 points.  Because the problem does not specify point values, the total factor of all its question prompts becomes the point value, which is $4$ points, because each question prompt has a default factor of $1.0$.

\begin{lstlisting}
\begin{problem}
For each of the following recurrences provide a closed form solution.

\ask 
$W(n) = W(n/2) + n$

\solfin $O($ \fin{n\lg{n}} $)$

\ask 
$W(n) = W(n/2) + n$

\solfin $o($ \fin{n^2} $)$


\ask 
$W(n) = W(n/2) + n$

\solfin $\Omega($ \fin{n\lg{n}} $)$

\ask 
$W(n) = W(n/2) + n$

\solfin $\omega($ \fin{n} $)$

\end{problem}
\end{lstlisting}
\end{example}

\begin{example}[Multiple Choice and  Free Response Questions]

Any number of question-solution prompts may be attached to an atom.
The example below starts with a free-response question and continues
with a multiple-choice question.
%
In a multiple choice question, use \lstinline`*` to indicate correct choices.
%
Because the first question has double the factor of the second, the
first solution box is worth $6.0$ points and the second question is
worth $3.0$ points.

\begin{lstlisting}
\begin{problem}[9.]

\ask[2.0] Give a closed-form solution in terms of $\Theta$ for the recurrence:
\[
W(n)=2 W(n/2)+ n.
\]

\sol
$W(n)=\Theta(n\lg{n})$

\onechoice[1.0]
Indicate whether the recurrence is root dominated,  leaf dominated, or balanced.

\choice Root dominated
\choice Leaf dominated
\choice* Balanced

\end{problem}
\end{lstlisting}
\end{example}


\begin{example}[True-False Questions]

True-false questions are a special case of multiple choice questions
and can be written as follows.

\begin{lstlisting}
\begin{problem}

\asktf
There are less that 10000 tigers left in the wild.

\solt

\end{problem}


\begin{problem}

\asktf
Rock climbing is an olympic sort since 1990.

\solf

\end{problem}
\end{lstlisting}



\begin{lstlisting}
\begin{problem}

\anychoice
Select all the correct statements belowe.

\choice* $2 + 2 = 4$
\choice $6 \pmod 4 = 4$
\choice* $8 \pmod 3 = 2$
\choice* $8 \pdiv 5 = 3$
\choice $12 \pmod 4 = 2$
\end{problem}
\end{lstlisting}
\end{example}

\begin{example}[Multi-Answer Questions]

Some questions require selecting multiple answers.
As in muliple-choice question, use \lstinline`*` to indicate correct choices.

\begin{lstlisting}
\begin{problem}

\anychoice
Select all the correct statements belowe.

\choice* $2 + 2 = 4$
\choice $6 \pmod 4 = 4$
\choice* $8 \pmod 3 = 2$
\choice* $8 \pdiv 5 = 3$
\choice $12 \pmod 4 = 2$
\end{problem}
\end{lstlisting}
\end{example}

\begin{example}
The following example illustrates factor distributions between
questions and solutions.  Because the problem does not specify point
values it receives the total factor of $17$ as the point value.
\begin{lstlisting}
\begin{problem}[Tricky Factors]
Give a closed-form
solution in terms of $\Theta$ for the following recurrences.  Also, state
whether the recurrence is dominated at the root, the leaves, or
equally at all levels of the recurrence tree.

% This question prompt distributes 9 factors over 1.5 factors
\ask[9.]
Give closed form for  
$f(n) = 5f(n/5) + \Theta(n)$

\sol[0.5] % This prompt gets 3.0 factors (and points).
$\Theta (n \lg n)$, balanced.

\sol[1.0] % This prompt gets 6.0 factors  (and points).
$\Theta (n \lg n)$, balanced.

% This question prompt distributes 8.0 factors over 2.0 factors 
\onechoice[8.] 

% This prompt has the default factor of 1.0.
% Via distribution, it gets 4.0 factors  (and points). 
\choice* Balanced 
% This choice 1.6 factors  (and points).
\choice[0.4] Leaves Dominated  
% This choice gets 2.4 factors  (and points).
\choice[0.6] Root dominated   
\end{problem}
\end{lstlisting}
\end{example}


\begin{example}
Consider now the same example as above, but this time
the problem specifies a point value of $34.0$
\begin{lstlisting}
\begin{problem}[34.][Tricky Factors and Points]
Give a closed-form
solution in terms of $\Theta$ for the following recurrences.  Also, state
whether the recurrence is dominated at the root, the leaves, or
equally at all levels of the recurrence tree.

% This question prompt distributes 9 factors over 1.5 factors
\ask[9.]
Give closed form for  
$f(n) = 5f(n/5) + \Theta(n)$

\sol[0.5] % This prompt gets 3.0 factors and 6.0 points.
$\Theta (n \lg n)$, balanced.

\sol[1.0] % This prompt gets 6.0 factors and 12.0 points).
$\Theta (n \lg n)$, balanced.

% This question prompt distributes 8.0 factors over 2.0 factors 
\onechoice[8.] 

% This prompt has the default factor of 1.0.
% Via distribution, it gets 4.0 factors and 8.0 points. 
\choice* Balanced 
% This choice 1.6 factors and 3.2 points.
\choice[0.4] Leaves Dominated  
% This choice gets 2.4 factors and 4.8 points.
\choice[0.6] Root dominated   
\end{problem}
\end{lstlisting}
\end{example}

\section{Time Management}

Diderot manages exam time in a way that mirrors in-class exams.

\begin{gram}[Time Limit and Due Date]
Each submittable chapter may be assigned a \defn{time limit} that
specifies the duration of the test.  For example, a quiz may be 20
minutes long.

Each submittable chapter may also be assigned a \defn{due date} specifying a date and a time until which it must be completed. 
\end{gram}

\begin{gram}[Taking the Exam]
Once a submittable chapter is released, a student can create a submission and take the test.  During the test, the students will see the remaining time on their screen.  The remaining time is calculated for each student as minimum of the unused portion of the time limit, and time left until the due date of the chapter.  This means that if a students starts an exam 30 minutes later that it is scheduled time, they will have 30 minutes less time to complete the exam.

After the due date of a submittable chapter, the students may not take the exam and will not see its contents.

Diderot does not use the release date of a chapter to enforce that the students start only at that time.  The students can start as soon as the chapter is released.
\end{gram}

\section{Saving and Submitting an Exam}

\begin{gram}
Diderot saves a test-taker's exam frequently (every few seconds) to make sure that no work would be lost due to problem such as internet connectivity.  

At any time, the student can submit their exam by clicking the ``Submit'' button.

When the time expires, the most recent version of the exam is submitted automatically.
\end{gram}

\section{Accommodations}

\begin{gram}
Personal accommodations of the students are automatically taken into account when a student is taking the exam.  For example, if a student has 50\% extra-time accommodation, then Diderot will take this is into account when administering the exam to that student.  

The instructor can change accommodations for each student on the Users page. 
\end{gram}
