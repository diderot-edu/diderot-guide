\chapter{Diderot Command Line Interface}
\label{ch:cli}

Diderot offers a command line interface to simplify interaction with
the platform via the command line.
%
The main advantage of the CLI over the web interface is that it is possible to upload materials onto the platform more quickly.

\begin{important}[Sources and Contributions]

%The CLI is available on AFS, at the following path: TODO (rohany): add the path in later.

To download the CLI, please clone 
%
\href{https://github.com/diderot-edu/diderot-cli}{this github repository}
%
and follow the instructions in \lstinline`README.md`. 

The Diderot CLI is public. Feel free to contribute or open an issue if there
are any commands or utilities that you think would be useful to have.
%
Please send feedback by 
\href{https://github.com/diderot-edu/diderot-cli/issues}{opening an issue}.

This guide is public and it source are available 
\href{https://github.com/diderot-edu/diderot-guide}{in this repo}.
%
Please send feeback by \href{https://github.com/diderot-edu/diderot-guide/issues}{opening an issue}.
\end{important}


\section{Basic Usage}

Use the \verb|diderot_student| or \verb|diderot_admin| scripts to interface with the CLI.
%
For help about the CLI, use the \verb|-h| or \verb|--help| commands.

\subsection{Credential Management}
Credentials are passed to the CLI in one of two ways. 
%
The first is to simply pass your credentials via the CLI using the \verb|--username| and \verb|--password| flags on the CLI. 
%
A more convenient (and recommended) way is to use a \emph{credentials file}.
%
The credential file format is simply a text file with your username on the first line and your password on the second. 
%
Use the \verb|--credentials| argument to point the CLI towards a file containing your credentials.
%
For easier usage, the CLI automatically looks at the files \verb|~/private/.diderot/credentials| and \verb|~/.diderot/credentials |for a credentials file of this form. 
%
If this file exists, then the CLI will automatically log you in, and no credentials need to be explicitly provided to the CLI.
%

\begin{important}
Your credentials file must have only "owner can read and write" permissions. 
%
You can  run \verb|chmod 600 <credentials file>| to set the appropriate permissions.
\end{important}

\begin{example}
An example credential file is below.
\begin{verbatim}
MyUsername
MyPassword
\end{verbatim}
\end{example}

\begin{important}
If you are using a custom path (rather than the default path) for your credentials file, or passing your username and password explicitly, then you will need supply these arguments as the first in the command.
%
Specifically, the
%
\verb|--username|/\verb|--password| or \verb|--credentials| 
%
arguments must be passed to the CLI as the first argument, before
additional commands!

For example, this command is correct:
\begin{verbatim}
$ diderot_student --credentials Path/To/My/Creds list_courses
\end{verbatim}
  
This command is incorrect:
\begin{verbatim}
$ diderot_student list_courses --credentials Path/To/My/Creds
\end{verbatim}
\end{important}

\section{Student Interface}

\begin{itemize}
  \item \verb|list_courses|: List all available course labels that the user can access.
    Operations which require a course input must use the one of the labels returned from this command.
  \item \verb|list_assignments|: Lists all assignments for a course.
  \item \verb|download_assignment|: Downloads all handout/writeup files (if present) from a
    target course and assignment into the current directory.
  \item \verb|submit_assignment|: Submit a file to a given course and homework.
\end{itemize}

More details about each command are below.

\subsection{list\_courses}

\verb|list_courses| lists all course labels that the user has access to.
%
The labels returned from \verb|list_courses| must be used as arguments to all other
operations that require a course label.
%
It accepts 0 arguments.

Example Usage:
\begin{verbatim}
$ diderot_student list_courses
\end{verbatim}

\subsection{list\_assignments}

\verb|list_assignments| is used to list all available assignments for a particular course.
%
It accepts only 1 argument.
\begin{itemize}
  \item \verb|course|: Course label to list homeworks of.
\end{itemize}

Example Usage:
\begin{verbatim}
$  diderot_student list_assignments courselabel
\end{verbatim}

\subsection{download\_assignment}

\verb|download_assignment| is used to download handout files associated with an assignment.
%
It attempts to download all handout files into the current directory.
% 
It accepts 2 arguments.

\begin{itemize}
  \item \verb|course|: Course label that the homework belongs to.
  \item \verb|homework|: Name of the homework to download handouts of.
\end{itemize}

Example Usage:
\begin{verbatim}
$  diderot_student download_assignment courselabel homeworkname
\end{verbatim}

\subsection{submit\_assignment}

\verb|submit_assignment| is used to submit handin files to an assignment.
%
Please make sure to check your homework handouts to see what file types
your homework expects.
%
It accepts 3 arguments.

\begin{itemize}
  \item \verb|course|: Course label that the homework belongs to.
  \item \verb|homework|: Name of the homework to submit to.
  \item \verb|handin_path|: Path to the handin file.
\end{itemize}

Example Usage:
\begin{verbatim}
$  diderot_student submit_assignment courselabel homeworkname ~/Path/To/My/Handin
\end{verbatim}

\section{Admin Interface}

The admin CLI contains all the above commands, and expands the set to include
some commands relating to manipulation of books and assignments.

\begin{itemize}
  \item \verb|list_books|: List all books, can optionally filter by course.
  \item \verb|list_parts|: List all parts of a book.
  \item \verb|list_chapters|: List all chapters of a book.
  \item \verb|create_part|: List all parts of a book.
  \item \verb|create_chapter|: Create a chapter in a given part of a
    book. Optionally specify the new chapter's label and or title. If
    not, these values are filled in with a default.
  \item \verb|{release/unrelease}_chapter|: Release or unrelease an existing chapter.
  \item \verb|upload_chapter|: Update the content of an existing
    chapter. Upload either a standalone pdf, pdf of a slideshow, or an
    xml document and image attachments.  The image upload accepts
    filenames, globs, and folder names (folders are recursively
    traversed).
  \item \verb|update_assignment|: Update the files relating to an existing homework on Diderot.  Upload any of the autograder tar, autograder makefile, pdf writeup, and student handout.
\end{itemize}

\subsection{list\_books}

\verb|list_books| is used to list book labels that the user has access to.
%
The result from \verb|list_books| is used as the argument book label for commands
that require a book label.
%
It accepts 2 arguments.

\begin{itemize}
  \item \verb|course|: Course label that the target book belongs to.
  \item \verb|--all|: Optional flag to list all books regardless of course. Not to be used when specifying a course.
\end{itemize}

Example Usage:
\begin{verbatim}
$  diderot_admin list_books courselabel
$  diderot_admin list_books --all
\end{verbatim}

\subsection{list\_parts}

\verb|list_parts| is used list the parts and numbers within a target book.
It accepts 2 arguments.

\begin{itemize}
  \item \verb|course|: Course label that the target book belongs to.
  \item \verb|book|: Label of the target book.
\end{itemize}

Example Usage:
\begin{verbatim}
$  diderot_admin list_parts courselabel booklabel
\end{verbatim}

\subsection{list\_chapters}

\verb|list_chapters| is used list the chapters and numbers within a target book.
It accepts 2 arguments.

\begin{itemize}
  \item \verb|course|: Course label that the target book belongs to.
  \item \verb|book|: Label of the target book.
\end{itemize}

Example Usage:
\begin{verbatim}
$  diderot_admin list_chapters courselabel booklabel
\end{verbatim}

\subsection{create\_part}

\verb|create_part| is used to create a part within a target book. It accepts 5 arguments.

\begin{itemize}
  \item \verb|course|: Course label that the target book belongs to.
  \item \verb|book|: Label of the target book.
  \item \verb|title|: Title to give to the newly created part.
  \item \verb|number|: Number to give the newly created part.
  \item \verb|--label|: Optional label to give to the newly created part. It defaults to a random value if not provided.
\end{itemize}

The command will error out if there already exists a part within the book with the same number.

Example Usage:
\begin{verbatim}
$ diderot_admin create_part courselabel booklabel title number --label new-label
\end{verbatim}


\subsection{create\_chapter}

\verb|create_chapter| is used to create a chapter within a target book. It accepts 6 arguments.

\begin{itemize}
  \item \verb|course|: Course label that the target book belongs to.
  \item \verb|book|: Label of the target book.
  \item \verb|--part|: Number of the part to create the chapter within. This argument is not used when the target book \emph{is a booklet}.
  \item \verb|--number|: Number to give the newly created chapter.
  \item \verb|--title|: Optional title to give to the newly created chapter. It defaults to 'Chapter'.
  \item \verb|--label|: Optional label to give to the newly created chapter. It defaults to a random value if not provided.
\end{itemize}

The command will error out if there already exists a chapter within the book with the same number.

Example Usage:
\begin{verbatim}
$ diderot_admin create_chapter courselabel booklabel --part partnum \
  --number newnum --title newtitle --label newlabel
\end{verbatim}

\subsection{release/unrelease\_chapter}

\verb|release_chapter| and \verb|unrelease_chapter| are used to release or unrelease a chapter. They each accept 3 arguments.

\begin{itemize}
  \item \verb|course|: Course label to release a chapter in. 
  \item \verb|book|: Book label to release a chapter in.

  \item \verb|--chapter_number| or \verb|--chapter_label|: Use one of these arguments to specify which chapter to release. Use either the chapter's number within the book, or its label.
\end{itemize}

Example Usage:
\begin{verbatim}
$ diderot_admin release_chapter courselabel booklabel --chapter_label chapterlabel
$ diderot_admin unrelease_chapter courselabel booklabel --chapter_label chapterlabel
\end{verbatim}

\subsection{upload\_chapter}

\verb|upload_chapter| is used to upload PDF, slide, or xml/mlx content to a chapter.
%
It accepts quite a few arguments and combinations.

\begin{itemize}
  \item \verb|course|: Course label to upload chapter to.
  \item \verb|book|: Book label to upload a chapter to.

  \item \verb|--chapter_number| or \verb|--chapter_label|: Use one of these arguments to specify which chapter to upload content to. Use either the chapter's number within the book, or its label.

  \item \verb|--pdf| or \verb|--slides| or \verb|--xml|: Use one of these arguments, specifying the desired file type to upload. Note that your slides must be in a PDF format to use the \verb|--slides| argument. The \verb|--xml| argument accepts both XML and MLX file formats.

  \item \verb|--video_url|: Optional URL to a video to embed within the chapter. This argument can only be used with the \verb|--pdf| and \verb|--slides| arguments.

  \item \verb|--xml_pdf|: Optional path to a PDF version of the XML content for printing. This argument can only be used with the \verb|--xml| argument.

  \item \verb|--attach|: Optional list of paths to attachments to upload along with an XML chapter upload. This argument can only be used with the \verb|--xml| argument. The command accepts paths to individual files, globs, and folders. Folders are recursively traversed.
\end{itemize}

Example Usage:
\begin{verbatim}
$ diderot_admin upload_chapter courselabel bookname --chapter_number chapternumber \
  --pdf Path/to/Pdf --video_url someurl

$ diderot_admin upload_chapter courselabel bookname --chapter_number chapternumber \
  --slides Path/to/slides --video_url someurl

$ diderot_admin upload_chapter courselabel bookname --chapter_number chapternumber \
  --xml Path/to/Xml --xml_pdf Path/To/XmlPdf \
  --attach Path/To/File Path/To/Folder/ Path/With/Glob/*
\end{verbatim}

\subsection{Uploading Chapters with a Makefile}

\begin{gram}
The user can use several simple Makefile rules to upload chapters onto
Diderot without.  To this end, the user will need to include
\verb|resources/makefile-upload-to-diderot| Makefile recipe in their
Makefile and then use rules that look like this
\begin{verbatim}
overview: overview/overview.xml overview/overview.pdf
upload_overview: NO=1
upload_overview: FILE=overview/overview
upload_overview: overview upload_xml_pdf
\end{verbatim}

Here the variable \verb|NO| specifies the chapter number and \verb|FILE| specifies the xml and pdf file names, assumed to be \verb|FILE.xml| and \verb|FILE.pdf| respectively.  
%
For a book chapter, the part can be specified using \verb|PART_NO| variable.
%
For attachments, the \verb|ATTACH| variable may be used.
%
The user can specify the course and the book being uploaded into by using the
\verb|LABEL_COURSE| and \verb|LABEL_TEXTBOOK| variables. 

Below is an example rule, where \verb|LABEL_COURSE| was set earlier in the Makefile.


\begin{verbatim}
upload_introduction_genome: TEXTBOOK="TB.VOL.1"
upload_introduction_genome: PART_NO=1 
upload_introduction_genome: NO=4
upload_introduction_genome: FILE=genome/genome
upload_introduction_genome: ATTACH=genome/media/*.jpg 
upload_introduction_genome: genome/genome.xml genome/genome.pdf upload_xml_pdf_attach
\end{verbatim}
\end{gram}

\begin{gram}[Example Makefile]
The \href{file://cli/attachments/Makefile-guide}{Makefile for the guide} uses this technique.

\end{gram}

\subsection{upload\_book}

\begin{gram}
\verb|upload_book| uploads a set of parts and chapters into a book or booklet.
%
It can be used to upload a complete booklet or book by creating all
the necessary parts and chapters specified in a json file.
%
It can also be used to upload any subset of parts and chapters.

It accepts two arguments.
%
\begin{itemize}
  \item \verb|course|: Course label that the book belongs to.
  \item \verb|book-data|: JSON file describing the parts and chapters to be uploaded (or the book/booklet).
\end{itemize}
\end{gram}

Example Usage:
\begin{verbatim}
diderot_admin upload_book  "CMU:Pittsburgh, PA:15210:Community:2020" volume-1.json
\end{verbatim}
%
 
\begin{gram}[JSON  for book uploads]
The JSON format required by book uploads is relatively self explanatory.
The file must specify the label of the book or booklet and then the parts (if applicable) and the chapters in the book. 
%
\begin{itemize}
\item Here is an example 
\href{file://cli/attachments/booklet-example.json}
{JSON file for a booklet}.
\item Here is an example JSON file 
\href{file://cli/attachment/book-example.json}
{JSON file for a book}.
\end{itemize}

Each chapter field must contain \verb|number| specifying the chapter
number and in addition may contain any subset of \verb|xml|,
\verb|xml_pdf|, \verb|pdf|, \verb|title|, \verb|label|.  Note that for
\verb|xml| chapters, the label specified in the chapter takes
precedence over specified in the json file if any.

\end{gram}



\subsection{update\_assignment}

\verb|update_assignment| is used to update files related to an assignment on Diderot.
%
It accepts 6 arguments.

\begin{itemize}
  \item \verb|course|: Course label that the assignment belongs to.
  \item \verb|homework|: Name of the assignment to update.
  \item \verb|--autograde-tar|: Optional path to the autograder tar file for the assignment. The autograde.tar file contains what is needed to run the assignment's autograder.
  \item \verb|--autograde-makefile|: Optional path to the autograde-Makefile for the assignment. The autograde-Makefile is used to execute the autograder on a student handin.
  \item \verb|--writeup|: Optional path to a PDF writeup of the assignment.
  \item \verb|--handout|: Optional path to a handout file that students can download to start the assignment.
\end{itemize}

Example Usage:
\begin{verbatim}
$ diderot_admin update_assignment courselabel homeworkname \
  --autograde-tar Path/To/Tar \
  --autograde-makefile Path/To/Makefile \
  --writeup Path/To/Writeup --handout Path/To/Handout
\end{verbatim}
