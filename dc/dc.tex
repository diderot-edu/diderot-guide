\chapter{DC: Diderot Compiler}
\label{mtl}
 

\section{Overview}

Diderot is an online book system that integrates discussions with
content.  Diderot consists of two separate systems that are
designed to work together:
%
\begin{itemize}
\item the Diderot website, which
provides the users (instructors and students) with an online interface
for reading books and holding discussions, 
\item the Diderot
compiler, \defn{DC} for short, that translates LaTeX and Markdown
sources to Diderot-style XML.
\end{itemize}
%
Diderot-style XML files may be uploaded onto the Diderot site. 
%
In addition to XML, Diderot site accepts conventional PDF documents
and slide decks for upload. 
%


\begin{important}[Sources and Contributions]
To go through the examples in this chapter, please clone 
%
\href{https://github.com/diderot-edu/guide}{this github repository}
%
and follow the instructions in \lstinline`INSTALL.md`. 

This guide is public.  Feel free to contribute to the sources and send feedback on Diderot.  
\end{important}

\begin{note}[Questions and Feedback]
You can click any ``mail''  button in this guide and send feedback or ask questions. 
\end{note}

\section{Typesetting with LaTex}

DC is compatible with LaTeX. Whatever updates you make to upload your notes on Diderot will be compatible with LaTeX. 
%
This means for example that you can generate a PDF document from your notes.
%
The reverse direction, going from LaTeX to Diderot, is not as definitive but works reasonably well.
%
If you have  LaTeX sources that you are able to compile and generate PDF from, then in most cases, you can use DC to generate XML from your LaTeX sources.  
%
DC also provides a few additional syntactic elements that helps improve the interactivity of the content.  
%

\begin{definition}[Books and Booklets]
After a LaTeX document is translated to XML via DC, it can be uploaded as a \defn{chapter} to a Diderot book or Diderot booklet.  
%
A Diderot \defn{book} is a collection of parts, where each \defn{part} is a collection of chapters.
%
A Diderot \defn{booklet} is a collection of chapters. 
\end{definition}

\begin{example}
The folders \lstinline`book` and \lstinline`booklet` contain a sample book (consisting of parts and chapters) and booklet (consisting only of chapters) respectively.
%
Each come with a  \lstinline`Makefile` for generating PDF and Diderot-XML from the sources. 
\end{example}

\begin{example}[PDF Generation]
You can generate the PDF for the whole book by running the following.
%
\begin{lstlisting}
$ cd examples/book
$ make book.pdf

$ cd examples/booklet
$ make book.pdf
\end{lstlisting}
\end{example}

\begin{example}[XML Generation]
For ease of publication on Diderot, XML generation occurs on a chapter by chapter basis.
%
You can generate the XML for  each chapter by specifying the chapter path.
\begin{lstlisting}
$ cd examples/book
$ make probability/expectation.xml

$ cd examples/booklet
$ make skyline/main.xml
\end{lstlisting}
\end{example}


\subsection{Basic Structure of LaTeX} 

As with Latex, DC requires a book to be organized into chapters, each of which then contains sections (section, subsection, subsubsection, and paragraph).  Each chapter and section can in turn contain ``elements'' each of which is an ``atom'' or a  ``group''.


\begin{example}[Sections]

A chapter is typically structured into a number of sections.

\begin{lstlisting}
\chapter{Introduction}
\label{ch:example}  % Chapters must have a label.
   
<elements>

\section{A Section}
\label{sec:example} % Optional but recommended section label.   

<elements>

\subsection{A Subsection}
\label{sec:example::sub} % Optional but recommended section label.   

<elements>

\subsubsection {A Subsubsection}
\label{sec:example::subsub}
<elements>

\paragraph {A paragraph}
\label{sec:example::paragraph}
<elements>

\end{lstlisting}

The term \lstinline`element` in the example refers to an "atom" or a "group".
\end{example}



\subsection{Atoms}
\label{sec:mtl::atoms}

\begin{definition}[Atom]
An \defn{atom} is either
\begin{enumerate}
\item a plain paragraph, or
\item a single-standing environment of the form

\begin{lstlisting}
\begin{<atom>}[Optional title]
\label{atom-label} % optional but recommended label
<atom body>
\end{<atom>}
\end{lstlisting}
%
The term \lstinline`<atom>` above can be replaced with any of the following:
\begin{itemize}
\item \lstinline`algorithm`, \lstinline`assumption`,
\item \lstinline`code`, \lstinline`corollary`, \lstinline`costspec`,
\item \lstinline`datastr` (data structure), \lstinline`datatype`, \lstinline`definition`
\item \lstinline`example`, \lstinline`exercise`,
\item \lstinline`gram`  (non descript atom, i.e., a paragraph),
\item \lstinline`hint`, 
\item \lstinline`important`, 
\item \lstinline`lemma`,
\item \lstinline`note`,
\item \lstinline`preamble` (as a  first atom of chapter), \lstinline`problem` (a problem for students to solve), \lstinline`proof`, \lstinline`proposition`,
\item \lstinline`remark` (an important note), \lstinline`reminder`,
\item \lstinline`solution` (a solution to an exercise/problem), \lstinline`syntax` (a piece of syntax)
\item \lstinline`task` (a task in an assignment), \lstinline`theorem`.
\end{itemize}
\end{enumerate}

\end{definition}

\begin{note}
Authors can currently use only these atoms defined above. We are working on allowing authors to define their own atoms.  
%
In the mean time, you can request new atoms  (please send feedback here).
\end{note}

\begin{important}
Atoms are \defn{single standing}, that is to say surrounded by "vertical
white spaces" or blank lines on both ends.
%
Therefore,  white space  matters. In the common case, this goes along with our intuition of how text is organized but is worth keeping in mind. For example, the following code will not be a definiton atom, but will be a plain paragraph atom, because definition is not single standing.

\begin{lstlisting}
We can now define Kleene closure as follows.
\begin{definition}
...
\end{definition}
\end{lstlisting}

The following is a definition atom, because it is single standing.
\begin{lstlisting}

\begin{definition}
...
\end{definition}

\end{lstlisting}
\end{important}

\begin{note}
Atoms can contain multiple paragraphs.  The following text consists of a text-paragraph atom and a definition atom.


\begin{lstlisting}
Paragraph 0. Sentence 1.
Sentence 2.

\begin{definition}[Definition Title]
Paragraph 1

Paragraph 2
\end{definition}
\end{lstlisting}
\end{note}

\subsubsection{Controlling granularity}

DC will create an atom for each text paragraph.  If a piece of text contains many small (one or two line) paragraphs, it can distracting for the user.  For example, the following text consists of three small paragraphs.
%
\begin{lstlisting}
If this then that.

If that then this.

This if and only of if that.
\end{lstlisting}

We usually don't write like this but sometimes it happens.
%
In such a case, you may want to wrap this text into a single paragraph atom.
%
\begin{lstlisting}
\begin{gram}[If and only If]
If this then that.

If that then this.

This if and only of if that.
\end{gram}
\end{lstlisting}
%
Alternatively you can wrap the text by curly braces as follows.
%
\begin{lstlisting}
{
If this then that.

If that then this.

This if and only of if that.
}
\end{lstlisting}
%
%
Both will have no impact on the PDF but on Diderot, you will have only one atom for the three sentences.

\subsection{Groups}
\label{sec:mtl::groups}

\begin{definition}[Group]
A \defn{group} consist of a sequence of atoms.  DC currently support only one kind of group: \lstinline`flex`.  On Diderot, a \lstinline`flex` will display its first atom and allow the user to reveal the rest of the atoms by using a simple switch.  This, for example, can be useful for hiding simple examples for a definition, the solution to an exercise, and a tangential remark, etc. 

\begin{lstlisting}
\begin{flex}
\begin{definition}[A Definition]
\label{def:a}

A definition
\end{definition}

\begin{example}[Simple Example I]
\label{ex:a-simple}
Simple example 1
\end{example}

\begin{example}[Simple Example II]
\label{ex:a-simple-2}
Simple example 2
\end{example}

\end{flex}
\end{lstlisting}
\end{definition}

\begin{note}
This has turned out be a favorite feature of Diderot for authors and students alike. 
%
You can see how this `flex` example works below, where a definition atom
has been paired flexibly with two examples.  Click the drawer icon
at the bottom left to open the `flex` and click again to close it.
\end{note}

\begin{flex}
\begin{definition}[A Definition]
\label{def:a}
A definition.
\end{definition}

\begin{example}[Simple Example]
\label{ex:simple-1}
Simple example 1
\end{example}

\begin{example}[Simple Example]
\label{ex:simple-2}
Simple example 2
\end{example}

\end{flex}
  

\subsection{Labels and References}

Diderot uses labels to identify atoms uniquely. All labels
in a book must be unique.  To help in authoring, we recommended giving
each chapter a unique label, and prepending each label with that of
the chapter.


\begin{example}

We recommend labeling content as follows.

\begin{lstlisting}
\chapter{Introduction}
\label{ch:intro}

\begin{preamble}
\label{prml:intro}
...
\end{preamble}

\section{Overview}
\label{sec:intro::overview}


This is a paragraph (atom) without a label.  The following paragraph (atom) has a label.


\begin{gram}
\label{grm:intro::present}
In this  section, we present...
\end{gram}

Here is another paragraph atom, consisting of two environments:
\begin{itemize}
...
\end{itemize}
\begin{enumerate}
...
\end{enumerate}

\end{lstlisting}
\end{example}

\begin{gram}[References]
To reference a label you can either use
\begin{itemize}
\item \lstinline`\href{label}{ref text}`
\item \lstinline`\ref{label}`.
\end{itemize}
%
DC replaces the former with \lstinline`\hyperref[][]` command so that we can get proper linked refs is latex/ pdf.
\end{gram}

DC generates a unique label for each atom, group, and section, in a document.
%
When doing so, DC uses different prefixes for labels: \lstinline`sec` for all sections, \lstinline`grp` for groups, and the following for atoms.   Atoms and their labels are shown below.
%
\begin{lstlisting}
algorithm : "alg"
assumption : "asm"
code : "cd"
corollary : "crl"
costspec : "cst"
datastr : "dtstr"
datatype : "adt"
definition : "def"
example : "xmpl"
exercise : "xrcs"
hint : "hint"
important : "imp"
lemma : "lem"
note : "nt"
gram : "grm"
preamble : "prmbl"
problem : "prb"
proof : "prf"
proposition : "prop"
remark : "rmrk"
reminder : "rmdr"
slide : "slide"
solution : "sol"
syntax : "syn"
task : "tsk"
theorem : "thm"
\end{lstlisting}

\subsection{Code}
\label{sec:mtl::code}
For code, you can use \lstinline`\lstinline` and the \lstinline`lstlisting` environment.  The language has to be specified first (see below for an example).  The Kate language highligting spec should be included in the "meta" directory and the name of the file should match that of the language.  For example if \lstinline`language = C`, then the Kate file should be \lstinline`meta/C.xml`.  If the language is a dialect, then, e.g., \lstinline`language = \{[Cdialect]C}`, then the file should be called \lstinline`CdialectC`.  
%
Kate highlighting definitions for most languages are available online.
%
If you need a custom language, you can probably write one with a bit of effort by starting with the Kate specification for a  suitably close language.

\begin{flex}
\begin{example}[Python Code]
Specify language as python, and include  line-numbers on the left of each line.
Don't forget to include the Kate highlighting file for python in the meta directory.

\begin{verbatim}
\begin{lstlisting}[language = python, numbers = left]
def is_even (i):
  if i %2 = 0:
    return true
  else 
    return false
\end{lstlisting}
\end{verbatim}
\end{example}

The code above will render like this:
%
\begin{lstlisting}[language = python, numbers = left]
def is_even (i):
  if i %2 = 0:
    return true
  else 
    return false
\end{lstlisting}
\end{flex}


\begin{example}[Code in C Dialect]
Specify the dialect preferred, and don't include line-numbers.
%
Don't forget to include the Kate highlighting file for the dialect in the meta directory.

\begin{verbatim}
\begin{lstlisting}[language = {[C0]C}]
main () {
  return void
}
\end{lstlisting}
\end{verbatim}
\end{example}
\subsection{Images}

You can include JPEG or PNG images by using the usual \lstinline`includegraphics` command.  
%
We currently don't support PDF images (will be available shortly).

There is one point to be careful about: sizing.  
%
In principle, you could use fixed sizes, e.g.,
\begin{lstlisting}
\includegraphics[width=5in]{myimage.jpg}
\end{lstlisting}
%
The problem with this approach is that the PDF output and the Diderot output will likely have different formats and the image that looks just fine on paper might look too big on Diderot or possibly vice versa.
%
I therefore recommend using relative widths using the following approach
\begin{lstlisting}
\includegraphics[width=0.5\textwidth]{myimage.jpg}
\end{lstlisting}
DC will translate this to 50\% width in html and 50\% of textwidth in PDF and the image will look consistent with its environment in both cases.  

\begin{note}
For PDF output, the author could also use \lstinline`\textheight`.
%
But this is not meaningful for HTML output, and will not work as expected.
\end{note}

\begin{important}[Scale is not Supported]
For PDF output it is also possible to use \lstinline`scale`.  For example,
\begin{lstlisting}
\includegraphics[scale=0.5]{myimage.jpg}
\end{lstlisting}

This means that the image should be shown at 50\% of its actual dimensions.  This has  the same issues as the absolute measures approach. Furthermore, DC doesn't detect attempt to detect the actual dimensions of the image and this approach is not supported.
\end{important}

\subsection{Typesetting Mathematics}

Diderot uses MathJax to render math environments.  This works in many cases, especially for use that is consistent with AMS Math packages.  There are a few important caveats. 
%
\begin{itemize}
\item Once you switch to math, try to stay in math.  You can switch to text mode using \lstinline`\mbox` but if you use macros inside  the \lstinline`mbox`, then they might not work (because mathjax doesn't know about your macros).  For example, the following won't work. 
\begin{lstlisting}
$\lstinline`xyz`$
\end{lstlisting}

\item The "tabular" environment does not work in MathJax.  Use "array" instead.

\item  The environment 
\begin{lstlisting}
\begin{alignat} 
... 
\end{alignat}
\end{lstlisting}
%
should be wrapped with \lstinline`\htmlmath`, e.g.,
%
\begin{lstlisting}
\htmlmath{
\begin{alignat} 
... 
\end{alignat}
}
\end{lstlisting} 
\end{itemize}


\subsection{Indexing}

\begin{flex}
In the near term, we plan to index books automatically by doing some natural-language processing.  
%
To aid this, we recommend marking definitional terms by using one of the following LaTeX commands: \lstinline`defb, \defe, \defn`.  You can define these commands as follows or any other way you prefer.
%
\begin{lstlisting}
\newcommand{\defb}[1]{\textbf{#1}}
\newcommand{\defe}[1]{\emph{#1}}
\newcommand{\defn}[1]{\textbf{\emph{#1}}}
\end{lstlisting}
%
The indexing algorithm will lookup these commands and link the use of these terms to their definition.

\begin{example}
\begin{definition}[Algorithm]
\label{def:algorithm}
An \defn{algorithm} is a recipe for solving a problem.
\end{definition}


\begin{definition}[Single Source Shortest Paths Problem]
\label{def:sssp}
The \defn{single source shortest path problem} or \defn{SSSP problem} for short requires finding  the shortest paths from a given source vertex to each and every vertex in a graph. 
\end{definition}

Given these definitions, the following sentence will be processed so that the terms ``algorithm'' and ``SSSP'' are linked to the two definitions above.
%
Dijkstra's algorithm solves the SSSP problem.
\end{example}
\end{flex}

\subsection{Colors}

You can use colors as follows
\begin{lstlisting}
\textcolor{red}{my text}
\end{lstlisting}

\subsection{Limitations}
\label{sec:mtl::limitations}

LaTeX has grown rich\footnote{Some would argue ``too rich'' and has become ``too big to break''.} but many authors tend to use a small subset.  
%
DC appears to work well for most uses. 
%
Here is a list of known limitations.  
\begin{itemize}

\item For XML translation work, the chapter should be compileable to PDF.

\item Do not use \lstinline`\input` directives in your chapters.

\item Each chapter must have a unique label.

\item Fancy packages will not work.  Stick to basic latex and AMS Math packages.

\item Support for tabular environment is limited: borders will not show and columns will be centered.  You can use the array (math/mathjax) as a substitute.  This could require using \lstinline`\mbox` for text fields.  
 
\item Center environment definitions will be ignored.

\item You can use \lstinline`itemize` and \lstinline`enumerate` in their basic form.  Changing the label format with \lstinline`enumitem` package and similar packages do not work.  You can imitate these by using heading for your items.  

\item Labeling and references are limited to atoms.  You can label atoms and refer to them, but you cannot label codelines, items in lists, etc.

\end{itemize}

\begin{remark}
The above limitations all pertain to XML generation.  Because DC works on a plain LaTeX document, these don't apply to generating other outputs such as 
PDF.
\end{remark}  

\subsection{Compiling Latex to XML}

You can translate LaTeX to XML by using the \lstinline`dc` binary.
%
The following instructions are tested on Mac OS X and Ubuntu.  The binaries in `bin` might not work on systems that are not Mac or Linux/Unix-like. 


As examples, 
the folders \lstinline`book` and \lstinline`booklet` contain several files.
%
\begin{itemize}
\item \lstinline`templates/diderot.sty`

   Supplies diderot definitions needed for compiling latex to pdf's.
   You don't need to modify this file.

\item \lstinline`templates/preamble.tex` 

   Supplies your macros that will be used by generating a pdf via pdflatex.  Nearly all packages and macros should be included here.  Each chapter will be compiled in the context of this file.  Ideally this file should
   - include as few packages as possible
   - define no environment definitions
   - macros should be simple

\item \lstinline`templates/preamble-diderot.tex` 

   Equivalent of preamble.tex but it is customized for XML output.  This usually means that most macros will remain the same but some will be simplified to work with `pandoc`.  If you don't need to customize, you can keep just one preamble.  The example in directory `booklet` does so.
\end{itemize}    

For ease of compilation and publishing on Diderot, we recommend
structuring your book sources to streamline your workflow for PDF
generation and for Diderot uploads.  We have found that the structure
outlined below.  The example book and booklet provided follow this
structure (see folders `book` and `booklet`).

\begin{gram}[Booklets]
 
 Booklets are books that don't have parts. For these  I recommend creating one directory per chapter and placing a single main.tex file to include all contain that you want.  Place all media (images, videos etc) under a media/ subdirectory. 
\begin{itemize}  
\item \lstinline`ch1/main.tex`
\item \lstinline`ch1/media/`: all my media files, *.png *.jgp, *.graffle, etc.
\item \lstinline`ch2/main.tex`
\item \lstinline`ch2/media/`: all my media files for chapter 2, *.png *.jgp, *.graffle, etc.
\item \lstinline`ch3/main.tex`
\end{itemize}
\end{gram}

\begin{gram}[Books]

Books have parts and chapters. We recommend structuring these as follows, where `ch1, ch2` etc can be replaced with names of your choice.
%
\begin{itemize}
\item \lstinline`part1/ch1.tex`
\item \lstinline`part1/ch2.tex`
\item \lstinline`part1/media-ch1/`
\item \lstinline`part1/media-ch2/`
\item \lstinline`part2/ch3.tex`
\item \lstinline`part2/ch4.tex`
\item \lstinline`part2/ch5.tex`
\item \lstinline`part2/media-ch3/`
\item \lstinline`part2/media-ch4/`
\item \lstinline`part2/media-ch5/`
\end{itemize}
   
\end{gram}


\begin{gram}[Making PDFs]
You can use \lstinline`pdflatex` to generate a PDF output.  See the \lstinline`Makefile` in book or booklet as examples.
%
For example, you can  invoke the \lstinline`Makefile` as follows to make a PDF:
\begin{lstlisting}
$ make book.pdf
\end{lstlisting}

\subsection{Making PDF for a Specific Chapter}
To make specific chapters,  extend the Makefile to compile each chapter separately.  See the \lstinline`Makefile` in book or booklet as examples.
%
For example, you can compile the chapter \lstinline`probability` in the \lstinline`book` directory as follows.
\begin{lstlisting}
$ make ch2
\end{lstlisting}
\end{gram}

\begin{gram}[Making XML]

To generate xml, simply use the Makefile provided as follows.
%
\begin{lstlisting}
$ make ch2/main.xml
\end{lstlisting}
%

To obtain more information from a run, you can turn on the verbose option.
%
\begin{lstlisting}
$ make verbose=true ch2/main.xml
\end{lstlisting}
\end{gram}

\begin{gram}
Error messages from the XML translator are not particularly useful at this time.  But, if you are able to generate a PDF, then you should be able to generate an XML. If you encounter a puzzling error try the "debug" version which will print the atoms as it goes through the input.
%
\begin{lstlisting}
$ make debug=true ch2/main.xml
\end{lstlisting}

If that fails, there is another relatively simply way to debug, see \href{sec:dc::internals}{Section Internals} for additional information.
\end{gram}

\begin{gram}[Using DC directly]
Assuming that you structure your book as suggested above, then you will mostly be using the Makefile but you could also use the DC tools directly. 
%
This tools translates the given input LaTeX file to xml.

\begin{lstlisting}
$ dc  -meta ./meta -preamble preamble.tex input.tex -o output.xml
\end{lstlisting}

The meta folder contains some files that may be used in the xml translation.  You can ignore this directory to start with and then start populating it based on your needs.  The main file that you might want to add are Kate highlighting specifications to be used for highlighting code.
\end{gram}

\begin{gram}[dc.dbg] 
This tools is the "debug" version of the \lstinline`dc` binary above. As you might notice, \lstinline`dc` doesn't currenty give reasonable error messages.  The debug version prints out the text that it parses, so you can have some sense of where things have gone wrong. 

\begin{lstlisting}
dc -meta ./meta -preamble preamble.tex input.tex -o output.xml
\end{lstlisting}
\end{gram}

%% ## Tool: tex2tex
%% This tools reads in your LaTeX sources, parses them, and writes it back.  It drops comments and normalized the whitespace but should otherwise return back a LaTeX file that is essentially the same as the input file.   You should not need to use this binary, which is primarily used for testing during development.

%% Examples: 
%% {lstlisting}
%% $ bin/tex2tex ./graph-contraction/star.tex -o ./s.tex
%% $ diff ./graph-contraction/star.tex ./s.tex
%% {lstlisting}
%% ### Tool: texel
%% This tool "normalizes" your latex sources.  This means that it

%% * atomizes your code, wrapping each paragraph into a non-descript "gram" atom if it is not already wrapped.

%% * wraps each atom by a "group", if not already wrapped by one.

%% * gives each segment (section, subsection, subsubsection, paragraph, atom) of the input file a label and it wraps each atom into a "group" if it is not already in a group.  A group is one of "cluster" "flex" "mproblem" (multipart problem).  

%% Generated labels have the form 
%% {lstlisting}
%% kind_prefix:chapter_label:segment_label
%% {lstlisting}
%% Here kind_prefix could for exmaple be
%% * `sec`, for section, subsection, subsubsection, paragraph
%% * `xmpl`, `thm`, for an example or a theorem.

%% The chapter_label is extracted from the chapter label given.  For exmaple, if the label has any one of the form 
%% {lstlisting}
%% ch:star | chapter:star | ch_star | ch__star | ch:_star | chap:_star
%% {lstlisting}
%% chapter_label will be `star`.

%% The tool takes the label, split it at the delimiters [:_]+ and if the prefix starts with "ch" it take the rest of the label as the chapter label.
 
%% Some example full labels:
%% * xmpl:star:simpleexample
%% * thm:star:costbound


\section{Typesetting with Markdown}

DC has basic support for Markdown to XML translation.  

\begin{gram}[Header]
For translation to work Markdown documents should have either of the following two headers
\begin{itemize}
\item Without front matter
\begin{lstlisting}
# Chapter Title
...
\end{lstlisting}  

\item  With YAML frontmatter.
\begin{lstlisting}
--- 
Frontmatter
---
# Chapter Title
...
\end{lstlisting}  
\end{itemize}
\end{gram}

\begin{important}[Special Symbols]
Special symbols should be escaped.
\begin{itemize}
\item The hash sign \# is a special symbol in LaTeX and must be escaped when using LaTeX macros.
\item In math mode \% is a LaTeX-Math special symbol and should be escaped.
\end{itemize}
\end{important}

\subsection{Chapters and Sections}

We support the atx-sytle headers, which start by 1-6 hash (\#) signs, e.g.,
\begin{lstlisting}
# This is a chapter

## This is a section

### This is a subsection

#### This is a subsubsection

##### This is a paragraph
\end{lstlisting}



\subsection{Code}
\label{sec:dc::md::code}

Usual Markdown conventions apply.  You can include inline code by using the backtick symbol. For code blocks use three or more backticks and specify the lanugage. It is important to finish a code block with the same number of backtics.

\begin{example}
\begin{lstlisting}
```html
<!DOCTYPE html>
<html lang="en">
<head>
    <meta charset="utf-8"/>
    <meta content="IE=edge" http-equiv="X-UA-Compatible"/>
    <meta content="width=device-width, initial-scale=1.0" name="viewport"/>
...

```

```ocaml
let x = 2 in
2+2
```

```python
def fun x:
  if x then x
  else x
```
\end{lstlisting}
\end{example} 

\subsection{Typesetting Mathematics}

You can use the usual LaTeX delimiters to include mathematics, i.e., for inline style, use single dollar sign.
%
For math blocks use double-dollars or \lstinline`\[`.

\subsection{Images}

\begin{flex}
\begin{gram}
You can include JPEG, PNG, and SVG images by using following syntax.
\begin{lstlisting}
![caption](image_file.ext){width=xy%}
\end{lstlisting}
\end{gram}

\begin{example}
The following code displays the image \lstinline`papers.svg`.
\begin{lstlisting}
![A categorization of CS papers.](papers.svg){width=90%}
\end{lstlisting}
\end{example}
\end{flex}

\subsection{Limitations}

Markdown to XML translation is relatively basic at this time.  If you need a particular feature not supported here, please let us know.
%
Here are few known limitations.
\begin{itemize}
\item Markdown usually allows blocks to be ended with equal or greater number of delimiter symbols than the start. For example a codeblock that starts with \lstinline!```! can be ended with \lstinline!```````!.  DC insists on blocks to start and end with the same length delimeters.

\item Blank lines in DC will always start an atom.  For example if you pepper blank lines into your list, each list item will turn into a separate atom.  This is probably not what you have intended.  
\begin{lstlisting}

* Item 1 is its own atom, because it starts with an blank line.

* Item 2 is its own atom, because it starts with an blank line.

\end{lstlisting}

\item DC does not allow nested paragraphs as some Markdown extensions do.  

\begin{lstlisting}

* Item 1 is its own atom, because it starts with an blank line.

    Here is a paragraph that you migh want to nest inside Item 1.  This will instead start a new atom in DC.  

* Item 2 is its own atom, because it starts with an blank line.

\end{lstlisting}

\end{itemize}


\subsection{Compiling Markdown To XML}

The guide includes a very simple markdown document\lstinline`markdown/example.md`.

\begin{gram}[Making XML]
To generate xml, simply use the Makefile provided as follows.
%
\begin{lstlisting}
$ make markdown/example.xml
\end{lstlisting}
%
If you would like to know more about how thing are being handled, you can turn on the verbose option
%
\begin{lstlisting}
$ make verbose=true markdown/example.xml 
\end{lstlisting}
\end{gram}

\begin{gram}[Using DC directly]
You could also use the DC tools directly as follows.
%
The command will generate  \lstinline`markdown/example.xml`.

\begin{lstlisting}
$ dc markdown/example.md
\end{lstlisting}

\end{gram}

\begin{gram}[The ``Debugger''] 
The tool \lstinline`dc.dbg` is the "debug" version of the \lstinline`dc` binary above. As you might notice, \lstinline`dc` doesn't currenty give reasonable error messages.  The debug version prints out the text that it parses, so you can have some sense of where things have gone wrong. 
See also \href{sec:dc::internals}{Section Internals} for additional debugging help.
\begin{lstlisting}
dc.dbg input.md -o output.xml
\end{lstlisting}
\end{gram}

\section{Internals}
\label{sec:dc::internals}

When using DC, a high-level understanding of the DC's internal could help.
%
DC works by splitting the LaTeX or Markdown document into atoms, which it treats as units.
%
It uses a Latex to HTML/XML or Latex to HTML/XML compiler to translate each atom and stitches the outputs back to construct the XML for the document.
%
The current implementation works with \href{https://www.pandoc.org}{pandoc}, though other compilers such as \href{https://dlmf.nist.gov/LaTeXML/}{LaTeXML} were also used at some point in the past.

When you receive an error, it will likely look like this:
\begin{lstlisting}
*latex_file_to_html: Executing command: pandoc --mathjax --lua-filter ./meta/codeblock.lua --lua-filter ./meta/span.lua /tmp/67.tex -o /tmp/67.html
Error in html translation.
Command exited with code: 65
Now exiting.make: *** [dc/dc.xml] Error 65
\end{lstlisting} 
% 
The error occurred when compiling atom \# 67, which is stored in the file \lstinline`/tmp/67.tex`.
%
To understand the error you can open this file.  The file is a complete LaTeX file and you can edit it and run \lstinline`pandoc` on it, possible with the flag \lstinline`--verbose`, which will likely be helpful.
%
You can ask \lstinline`pandoc` to generate a standalone html, by passing the \lstinline`-s` flag, so that you can open the result and examine how it looks rrin your browser.
%
\begin{lstlisting}
pandoc --verbose --mathjax --lua-filter ./meta/codeblock.lua --lua-filter ./meta/span.lua /tmp/67.tex -s -o /tmp/67.html
\end{lstlisting}
%
Once you figure out what the problem is what remains is to update your main document and proceed.
%
Basically the same workflow applies to Markdown inputs.

This is all you need. Happy Diderowing!
