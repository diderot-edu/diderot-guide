\chapter{DC: Diderot Compiler}
\label{ch:dc}
 

\section{Overview}

Diderot is an online book system that integrates discussions with
content.  Diderot consists of two separate systems that are
designed to work together:
%
\begin{itemize}
\item the Diderot website, which
provides the users (instructors and students) with an online interface
for reading books and holding discussions, 
\item the Diderot
compiler, \defn{DC} for short, that translates LaTeX and Markdown
sources to Diderot-style XML.
\end{itemize}
%
Diderot-style XML files may be uploaded onto the Diderot site. 
%
In addition to XML, Diderot site accepts conventional PDF documents
and slide decks for upload. 
%

In this chapter, we briefly review typesetting for Diderot using LaTex and Markdown.
%
In \href{ch:publish}{the next chapter}, we discuss some important points about how to publish content on Diderot and how to structure your source code.

\begin{important}[Sources and Contributions]
To go through the examples in this chapter, please clone 
%
\href{https://github.com/diderot-edu/diderot-guide}{this github repository}
%
and follow the instructions in \lstinline`INSTALL.md`. 

This guide is public and it source are available 
\href{https://github.com/diderot-edu/diderot-guide}{in this repo}.
%
Please send feeback by \href{https://github.com/diderot-edu/diderot-guide/issues}{opening an issue}.
\end{important}

%% \begin{note}[Questions and Feedback]
%% You can click any ``mail''  button in this guide and send feedback or ask questions. 
%% \end{note}

\section{Typesetting with LaTex}

DC is compatible with LaTeX. Whatever updates you make to upload your notes on Diderot will remain compatible with LaTeX and  you can generate a PDF document from your notes.
%
The reverse direction, going from LaTeX to Diderot, is not as
definitive but works reasonably well and with some care you can
compile your LaTeX sources to XML.
%


\begin{definition}[Books and Booklets]
After a LaTeX document is translated to XML via DC, it can be uploaded as a \defn{chapter} to a Diderot book or Diderot booklet.  
%

\begin{itemize}
\item
A Diderot \defn{book} is a collection of parts, where each \defn{part} is a collection of chapters.

\item
A Diderot \defn{booklet} is a collection of chapters. 
\end{itemize}
\end{definition}

\begin{example}
In
\href{https://github.com/diderot-edu/diderot-guide}{guide sources},
%
the folders \lstinline`book` and \lstinline`booklet` contain a sample book (consisting of parts and chapters) and booklet (consisting only of chapters) respectively.
\end{example}

%% Each come with a  \lstinline`Makefile` for generating PDF and Diderot-XML from the sources. 
%% \begin{example}[PDF Generation]
%% You can generate the PDF and HTML for the whole book by running the following.
%% %
%% \begin{lstlisting}
%% $ cd examples/book
%% $ make pdf
%% $ make html

%% $ cd examples/booklet
%% $ make pdf
%% $ make html
%% \end{lstlisting}
%% \end{example}

%% \begin{example}[XML Generation]
%% For ease of publication on Diderot, XML generation occurs on a chapter by chapter basis.
%% %
%% You can generate the XML for  each chapter by specifying the chapter path.
%% \begin{lstlisting}
%% $ cd examples/book
%% $ make probability/expectation.xml

%% $ cd examples/booklet
%% $ make skyline/main.xml
%% \end{lstlisting}
%% \end{example}


\subsection{Books, Chapters, and Sections} 

DC requires a book (or booklet) to be organized into chapters, each of which then contains sections (section, subsection, subsubsection, and paragraph).  Each chapter and section can in turn contain ``elements'' each of which is an ``atom'' or a  ``group''.


\begin{example}[Sections]

A chapter typically consists of a number of sections.

\begin{lstlisting}
\chapter{Introduction}
\label{ch:introduction}  % Chapters must have a label.
   
<elements>

\section{A Section}
\label{sec:introduction} % Optional but recommended section label.   

<elements>

\subsection{A Subsection}
\label{sec:introduction::sub} % Optional but recommended section label.   

<elements>

\subsubsection {A Subsubsection}
\label{sec:introduction::subsub}
<elements>

\paragraph {A paragraph}
\label{sec:introduction::paragraph}
<elements>

\end{lstlisting}

The term \lstinline`element` in the example refers to an "atom" or a "group".
\end{example}



\subsection{Atoms}
\label{sec:mtl::atoms}


\begin{definition}[Atom]
To increase interactivity, Diderot relies on atoms.
%
Atoms are smallest unit of sections and can be included in any section (i.e., section,  subsection, subsubsection, and paragraph).
%
In diderot, essentially all content  must be inside of atoms.
%
Like any other LaTeX section, an atom cannot be included inside of a
LaTeX environment.
%

An \defn{atom} is either
\begin{enumerate}
\item a plain paragraph, or
\item a named block of text of the form

\begin{lstlisting}
\begin{<atom>}[Optional title]
\label{atom-label} % optional but recommended label
<atom body>
\end{<atom>}
\end{lstlisting}
%
The term \lstinline`<atom>` above can be replaced with any of the following:
\begin{itemize}
\item \lstinline`algorithm`, \lstinline`answer`, \lstinline`assumption`,
\item \lstinline`code`, \lstinline`corollary`, \lstinline`costspec`,
\item \lstinline`datastr` (data structure), \lstinline`datatype`, \lstinline`definition`
\item \lstinline`example`, \lstinline`exercise`,
\item \lstinline`gram`  (non descript atom, i.e., a paragraph),
\item \lstinline`hint`, 
\item \lstinline`important`, 
\item \lstinline`lemma`,
\item \lstinline`note`,
\item \lstinline`observe`, \lstinline`observation`,
\item \lstinline`preamble` (as a  first atom of chapter), \lstinline`problem` (a problem for students to solve), \lstinline`proof`, \lstinline`proposition`,
\item \lstinline`question`, \lstinline`quote`,
\item \lstinline`remark` (an important note), \lstinline`reminder`,
\item \lstinline`solution` (a solution to an exercise/problem), \lstinline`syntax` (a piece of syntax)
\item \lstinline`task` (a task in an assignment), \lstinline`theorem`.
\end{itemize}
\end{enumerate}

\end{definition}

\begin{flex}

\begin{important}[Figure and Table Atoms]
In addition to the atoms above, Diderot also supports \lstinline`figure` and \lstinline`table` atoms, which have slightly different syntax:
\begin{lstlisting}
\begin{<atom>}[Optional title]
<atom body>
\caption{Caption}
\label{atom-label} % optional but recommended label
\end{<atom>}
\end{lstlisting}
\end{important}


\begin{example}
The following LaTeX code for a table atom generates the atom shown below.

\begin{lstlisting}
\begin{table}
\label{tbl:dc::motor-efficiency}
\begin{tabular}{cc}
Type & Efficiency
\\
Internal combustion Engine & \%20
\\
Electric Motor & \%90
\end{tabular}
\caption{Energy efficiency of internal-combustion engines are significantly lower than those of electric motors.}
\end{table}
\end{lstlisting}
\end{example}

\begin{table}
\label{tbl:dc::motor-efficiency}
\begin{tabular}{cc}
Type & Efficiency
\\
Internal combustion Engine & \%20
\\
Electric Motor & \%90
\end{tabular}
\caption{Energy efficiency of  internal-combustion engines are significantly lower than those of electric motors.}
\end{table}
\end{flex}

\begin{important}[Nesting of Atoms]
A named atom cannot be nested inside another named atom.
%
Similarly, a named atom cannot be nested inside a LaTeX environment.

For example, the following is wrong:
\begin{lstlisting}
\begin{itemize}
\item 
\begin{gram}[This is a titled paragraph]
But it is wrong, because it includes an atom inside of a LaTeX environment.
\end{gram}
\end{itemize}
\end{lstlisting}
\end{important}

\begin{note}
While an atom can be a paragraph, not every paragraph is an atom.  Paragraphs only at the ``top level'', not inside LaTeX environments are atoms. 
\end{note}

\begin{note}
Authors can currently use only the atoms defined above. We are working on allowing authors to define their own atoms.  
%
In the mean time, you can request new atoms by 
\href{mailto:umut@cs.cmu.edu}{emailing Umut}.
\end{note}

\begin{important}
Atoms are \defn{single standing}, that is to say surrounded by "vertical
white spaces" or blank lines on both ends (or LaTeX comments).
%
Therefore,  white space  matters. In the common case, this goes along with our intuition of how text is organized but is worth keeping in mind. For example, the following code is a plain paragraph atom but not a definition atom, because definition is not single standing.

\begin{lstlisting}
We can now define Kleene closure as follows.
\begin{definition}
...
\end{definition}
\end{lstlisting}

The following is a definition atom, because it is single standing.
\begin{lstlisting}

\begin{definition}
...
\end{definition}

\end{lstlisting}
\end{important}

\subsubsection{Controlling granularity}

DC will create an atom for each top-level paragraph (paragraph not nested inside LaTeX environments).  If a piece of text contains many small (one or two line) paragraphs, it can distracting for the user.  For example, the following text consists of three small paragraphs.
%
\begin{lstlisting}
If this then that.

If that then this.

This if and only of if that.
\end{lstlisting}

We usually don't write like this but sometimes it happens.
%
In such a case, you may want to wrap this text into a single paragraph atom.
%
\begin{lstlisting}
\begin{gram}[If and only If]
If this then that.

If that then this.

This if and only of if that.
\end{gram}
\end{lstlisting}
%
Alternatively you can wrap the text by curly braces as follows.
%
\begin{lstlisting}
{
If this then that.

If that then this.

This if and only of if that.
}
\end{lstlisting}
%
%
Both will have no impact on the PDF but on Diderot, you will have only one atom for the three sentences.


\subsection{Groups}
\label{sec:mtl::groups}

\begin{definition}[Group]
A \defn{group} consist of a sequence of atoms.  DC currently support only one kind of group: \lstinline`flex`.  On Diderot, a \lstinline`flex` will display its first atom and allow the user to reveal the rest of the atoms by using a simple switch.  This, for example, can be useful for hiding simple examples for a definition, the solution to an exercise, and a tangential remark, etc. 

\begin{lstlisting}
\begin{flex}
\begin{definition}[A Definition]
\label{def:a}

A definition
\end{definition}

\begin{example}[Simple Example I]
\label{xmpl:a-simple}
Simple example 1
\end{example}

\begin{example}[Simple Example II]
\label{xmpl:a-simple-2}
Simple example 2
\end{example}

\end{flex}
\end{lstlisting}
\end{definition}

\begin{note}
This has turned out be a favorite feature of Diderot for authors and students alike. 
%
You can see how this `flex` example works below, where a definition atom
has been paired flexibly with two examples.  Click the drawer icon
at the bottom left to open the `flex` and click again to close it.
\end{note}

\begin{flex}
\begin{definition}[A Definition]
\label{def:a}
A definition.
\end{definition}

\begin{example}[Simple Example]
\label{ex:simple-1}
Simple example 1
\end{example}

\begin{example}[Simple Example]
\label{ex:simple-2}
Simple example 2
\end{example}

\end{flex}
  

\subsection{Labels and References}
\label{sec:dc::labels-refs}
Diderot uses labels to identify atoms uniquely. All labels in a book
must be unique.  To help in authoring, we recommended giving each
chapter a unique label, and prepending each label in that chapter,
with the chapter label.


\begin{example}

We recommend labeling content as follows.

\begin{lstlisting}
\chapter{Introduction}
\label{ch:intro}

\begin{preamble}
\label{prmbl:intro}
...
\end{preamble}

\section{Overview}
\label{sec:intro::overview}


This is a paragraph (atom) without a label.  The following paragraph (atom) has a label.


\begin{gram}
\label{gr:intro::present}
In this  section, we present...
\end{gram}

Here is another paragraph atom, consisting of two environments:
\begin{itemize}
...
\end{itemize}
\begin{enumerate}
...
\end{enumerate}

\end{lstlisting}
\end{example}

\begin{gram}[References]
To reference a label you can either use
\begin{itemize}
\item \lstinline`\href{label}{ref text}`
\item \lstinline`\ref{label}`.
\end{itemize}
%
Note that the latter is compatible with PDF generation, whereas the former is not handled by \lstinline`pdflatex` as we would hope (generate a reference to the section).

Note that you can use \lstinline`\href{url}{ref text}` to generate a link to a url.
\end{gram}

DC generates a unique label for each atom, group, and section, in a document.
%
When doing so, DC uses different prefixes for labels: \lstinline`sec` for all sections, \lstinline`grp` for groups, and the following for atoms.   Atoms and their labels are shown below.
%
\begin{lstlisting}
algorithm : "alg"
assumption : "asm"
code : "cd"
corollary : "crl"
costspec : "cst"
datastr : "dtstr"
datatype : "adt"
definition : "def"
example : "xmpl"
exercise : "xrcs"
hint : "hint"
important : "imp"
lemma : "lem"
note : "nt"
gram : "grm"
observe : "obs"
observation : "obs"
preamble : "prmbl"
problem : "prb"
proof : "prf"
proposition : "prop"
quote : "quo"
remark : "rmrk"
reminder : "rmdr"
slide : "slide"
solution : "sol"
syntax : "syn"
task : "tsk"
theorem : "thm"
\end{lstlisting}


\begin{gram}[Limitations]
In LaTeX you can label pretty much anything. 
%
For example, you can label a line of code and refer to it via a number.
%
In DC, labeling is restricted to atoms, sections, and chapters, with the exception of equation labels, which are supported.
%
Learn more about the 
\href{sec:dc::typesetting-mathematics}{equation labels}.

\end{gram}

\subsection{Code}
\label{sec:mtl::code}
For code, you can use \lstinline`\lstinline` and the \lstinline`lstlisting` environment.  The language has to be specified first (see below for an example).  The Kate language highligting spec should be included in the "meta" directory and the name of the file should match that of the language.  For example if \lstinline`language = C`, then the Kate file should be \lstinline`meta/C.xml`.  If the language is a dialect, then, e.g., \lstinline`language = \{[Cdialect]C}`, then the file should be called \lstinline`CdialectC`.  
%
Kate highlighting definitions for most languages are available online.
%
If you need a custom language, you can probably write one with a bit of effort by starting with the Kate specification for a  suitably close language.

\begin{flex}
\begin{example}[Python Code]
Specify language as python, and include  line-numbers on the left of each line.
Don't forget to include the Kate highlighting file for python in the meta directory.

\begin{verbatim}
\begin{lstlisting}[language = python, numbers = left]
def is_even (i):
  if i %2 = 0:
    return true
  else 
    return false
\end{lstlisting}
\end{verbatim}
\end{example}

The code above will render like this:
%
\begin{lstlisting}[language = python, numbers = left]
def is_even (i):
  if i %2 = 0:
    return true
  else 
    return false
\end{lstlisting}
\end{flex}


\begin{example}[Code in C Dialect]
Specify the dialect preferred, and don't include line-numbers.
%
Don't forget to include the Kate highlighting file for the dialect in the meta directory.

\begin{verbatim}
\begin{lstlisting}[language = {[C0]C}]
main () {
  return void
}
\end{lstlisting}
\end{verbatim}
\end{example}
\subsection{Images}

You can include JPEG or PNG images by using the usual \lstinline`includegraphics` command.  
%
We don't support PDF images, because current PDF viewers (browser plugins) are rather clunky and could adversely impact user experience.
%
When including images, there are two points to be careful about: naming and sizing.
%

When naming the images, it is important the file name of each and every image in a chapter is unique.
%
Diderot indexes images based on their filenames and ignores their paths.
%
For example, it will treat
%
\lstinline`/tmp/image1.jpg` and  \lstinline`image1.jpg`
%
as the same .


When sizing your images, you could, in principle, use absolute sizes, e.g.,
\begin{lstlisting}
\includegraphics[width=5in]{myimage.jpg}
\end{lstlisting}
%
The problem with this approach is that the PDF output and the Diderot output will likely have different formats and the image that looks just fine on paper might look too big on Diderot or possibly vice versa.
%
We therefore recommend using relative widths using the following approach
\begin{lstlisting}
\includegraphics[width=0.5\textwidth]{myimage.jpg}
\end{lstlisting}
DC will translate this to 50\% width in html and 50\% of \lstinline`\textwidth` 
in PDF and the image will look consistent with its environment in both cases.  
%
When specifying the width, it is important to always include the fraction.  For example, if you want the width to be exactly equal to \lstinline`\textwidth` then use \lstinline`width=1.0\textwidth`.

\begin{note}
For PDF output, the author could also use \lstinline`\textheight`.
%
But this is not meaningful for HTML output, and will not work as expected.
\end{note}


\begin{important}[Scale is not Supported]
For PDF output it is also possible to use \lstinline`scale`.  For example,
\begin{lstlisting}
\includegraphics[scale=0.5]{myimage.jpg}
\end{lstlisting}

This means that the image should be shown at 50\% of its actual dimensions.  This has  the same issues as the absolute measures approach. Furthermore, DC doesn't detect attempt to detect the actual dimensions of the image and this approach is not supported.
\end{important}

\begin{gram}[Uploading Images]
When uploading a chapter, via the UI or via the CLI, the user needs to upload the images explicitly.   
\end{gram}

\subsection{Typesetting Mathematics}
\label{sec:dc::typesetting-mathematics}

Diderot uses MathJax to render math environments.  This works in many cases, especially for use that is consistent with AMS Math packages.  
%
A reasonably complete list of commands available in 
\href{http://www.onemathematicalcat.org/MathJaxDocumentation/TeXSyntax.htm#alphaList}{MathJax can be found here.}
%

There is one important caveat. 
%
Once you switch to math mode, try to stay in math mode.  You can switch to text mode using \lstinline`\text` but if you use macros inside  the \lstinline`\text`, then they might not work (because mathjax doesn't know about your macros).  For example, the following won't work. 
\begin{lstlisting}
$\lstinline`xyz`$
\end{lstlisting}



Diderot and DC support equation numbering via MathJax with one restriction: the reference to an equation must be made in the same chapter that the equation is defined in.
%
To refer to an equation label simply use the macro \lstinline`\eqref` or \lstinline`\ref` in math mode.  Note that the reference should be from within the math mode.

\begin{example}[Equation Labels]
The following LaTeX source defines an equation label and refers to it. 

\begin{lstlisting}
\begin{equation}
  \int_0^\infty \frac{x^3}{e^x-1}\,dx = \frac{\pi^4}{15}
  \label{eq:sample}
\end{equation}

Consider  Equation~$\eqref{eq:sample}$ shown above.
\end{lstlisting}

It will be rendered as follows.

\begin{equation}
  \int_0^\infty \frac{x^3}{e^x-1}\,dx = \frac{\pi^4}{15}
  \label{eq:sample}
\end{equation}

Consider Equation~$\eqref{eq:sample}$ shown above.


\end{example}

\begin{example}[Multiple Equations]
Other environments such as \lstinline`align` may be used to typeset multiple equations and label them. 
%
As an example, consider the following LaTeX code.

\begin{lstlisting}
\begin{align}
  \int_0^\infty \frac{x^3}{e^x-1}\,dx = \frac{\pi^4}{15}
  \label{eq:sample-2}
\\
  \int_0^\infty \frac{x^3}{e^x-1}\,dx = \frac{\pi^4}{15}
  \label{eq:sample-3}
\end{align}

Equation~$\eqref{eq:sample-3}$ follows from Equation~$\eqref{eq:sample-2}$.  And furthermore, Equation~$\eqref{eq:sample-2}$ follows from Equation~$\eqref{eq:sample-3}$.  
\end{lstlisting}

This will render on Diderot as follows.

\begin{align}
  \int_0^\infty \frac{x^3}{e^x-1}\,dx = \frac{\pi^4}{15}
  \label{eq:sample-2}
\\
  \int_0^\infty \frac{x^3}{e^x-1}\,dx = \frac{\pi^4}{15}
  \label{eq:sample-3}
\end{align}

Equation~$\eqref{eq:sample-3}$ follows from Equation~$\eqref{eq:sample-2}$.  And furthermore, Equation~$\eqref{eq:sample-2}$ follows from Equation~$\eqref{eq:sample-3}$.  
\end{example}

\subsection{Indexing}

\begin{flex}
In the near term, we plan to index books automatically by doing some natural-language processing.  
%
To aid this, we recommend marking definitional terms by using one of the following LaTeX commands: \lstinline`\defb, \defe, \defn`.  You can define these commands as follows or any other way you prefer.
%
\begin{lstlisting}
\newcommand{\defb}[1]{\textbf{#1}}
\newcommand{\defe}[1]{\emph{#1}}
\newcommand{\defn}[1]{\textbf{\emph{#1}}}
\end{lstlisting}
%
The indexing algorithm will look up these commands and link the use of these terms to their definition.

\begin{example}
\begin{definition}[Algorithm]
\label{def:algorithm}
An \defn{algorithm} is a recipe for solving a problem.
\end{definition}


\begin{definition}[Single Source Shortest Paths Problem]
\label{def:sssp}
The \defn{single source shortest path problem} or \defn{SSSP problem} for short requires finding  the shortest paths from a given source vertex to each and every vertex in a graph. 
\end{definition}

Given these definitions, the following sentence will be processed so that the terms ``algorithm'' and ``SSSP'' are linked to the two definitions above.
%
Dijkstra's algorithm solves the SSSP problem.
\end{example}
\end{flex}

\subsection{Colors}

You can use colors as follows
\begin{lstlisting}
\textcolor{red}{my text}
\end{lstlisting}


\subsection{Embedding Videos}

You can embed a video to your notes by using \lstinline`\video{url}{text}` command, where \lstinline`text` stands for a piece of text that will be displayed on browsers incapable of handling embedded video.
%
For example, the following code, when uploaded on diderot will embed the corresponding YouTube video.
%
\begin{lstlisting}
Here is a \video{https://www.youtube.com/embed/8QyE9D_3ezQ}{tango video.}
\end{lstlisting}

\begin{example}[Video Embedding]
The video embedding code will look like this.

Here is a tango video 
%
\video{https://www.youtube.com/embed/8QyE9D_3ezQ}{Murat and Michelle Erdemsel.}

\end{example}

\subsection{File Attachments}

You can attach a file to a chapter by using the command \lstinline`\href{file://path.name.ext}{referral text}` command.
%
When uploading your chapter, simply include the file \lstinline{name.ext} in your attachments.
%


\begin{example}[File Attachment]
The following code, when uploaded on diderot will create a link to the attached file (requires uploading the file \lstinline`dc.tex` as an attachment).  
%
\begin{lstlisting}
Here is the \href{file://dc.tex}{LaTeX source code} for this chapter of the guide.
\end{lstlisting}

Here is the \href{file://dc.tex}{LaTeX source code} for this chapter of the guide.
\end{example}


\begin{important}[Uniqueness of Attachments]
File name of each and every image and attachment in a chapter most be unique.
%
Diderot indexes images and attachments based on their filenames and ignores their paths.
%
For example, it will treat
%
\lstinline`/tmp/image1.jpg` and  \lstinline`image1.jpg` 
%
as the same.

\end{important}

\subsection{Limitations}
\label{sec:mtl::limitations}

LaTeX has grown rich (and perhaps too rich) but many authors tend to use a small subset.  
%
DC appears to work well for most uses. 
%
Here is a list of known limitations.  
\begin{itemize}

\item For XML translation work, the chapter should be compileable to PDF.  In general, recommended practice is to make sure that you can generate a PDF from your sources and then generate an XML.

\item Do not use \lstinline`\input` directives in your chapters but you can use them in preambles and packages included at the top of the LaTeX document.

\item All macro definitions via \lstinline`newcommand` are restricted to the ``preamble'' of the book.

\item Each chapter must have a unique label.

\item Fancy packages are not supported.  Stick to basic latex and AMS Math packages.

\item Support for tabular environment is limited: borders will not show and columns will be centered.  You can use the array (math/mathjax) as a substitute, and  use \lstinline`\mbox` for text fields.  
 
\item You can use \lstinline`itemize`, \lstinline`enumerate`, and \lstinline`description` environments in their basic form.  Changing the label format with \lstinline`enumitem` package and similar packages do not work.  You can imitate these by using heading for your items.  

\item Labeling and references are limited to chapters, sections, and atoms.  You can label these and refer to them.  Labels in other latex environments such as, items in lists, code lines, and equations are not supported.

\end{itemize}

\begin{remark}
The above limitations all pertain to XML generation.  Because DC works on a plain LaTeX document, these don't apply to generating other outputs such as 
PDF.
\end{remark}  



\section{Typesetting with Markdown}

DC has basic support for Markdown to XML translation.  

\begin{gram}[Header]
For translation to work Markdown documents should have either of the following two headers
\begin{itemize}
\item Without front matter
\begin{lstlisting}
# Chapter Title
...
\end{lstlisting}  

\item  With YAML frontmatter.
\begin{lstlisting}
--- 
Frontmatter
---
# Chapter Title
...
\end{lstlisting}  
\end{itemize}
\end{gram}

\begin{important}[Special Symbols and HTML code]
We have observed a couple of common mistakes in Markdown that can lead to problems publishing your work on Diderot.

First one concerns special symbols, which should be escaped.
\begin{itemize}
\item The hash sign \# is a special symbol in LaTeX and must be escaped when using LaTeX macros.
\item In math mode \% is a LaTeX-Math special symbol and should be escaped.
\end{itemize}

The second concerns html code.  You can use html tags in Markdown for various formatting purposes.  If you want to use html syntax for  any other reason, then please indicate that it is code.  For example, the following is an incorrect use of html code, because it doesn't close the tag \lstinline`table`:
\begin{lstlisting}
Here is an example html tag ``<table>''
\end{lstlisting}

The unclosed html tag here should be placed inside a code, e.g., by using back-tick delimiters. 
\begin{lstlisting}
Here is an example html tag `<table>`
\end{lstlisting}

\end{important}

\begin{important}[Chapter Titles]
In any given book, all chapter titles must be unique.  
%
This ensures that all chapters are uniquely labeled: \lstinline`dc` assigns a chapter with title ``This is a Chapter Title'' the label ``ch:this-is-a-chapter-title''.
\end{important}


\subsection{Chapters and Sections}

We support the atx-sytle headers, which start by 1-6 hash (\#) signs, e.g.,
\begin{lstlisting}
# This is a chapter

## This is a section

### This is a subsection

#### This is a subsubsection

##### This is a paragraph
\end{lstlisting}



\subsection{Code}
\label{sec:dc::md::code}

Usual Markdown conventions apply.  You can include inline code by using the backtick symbol. For code blocks use three or more backticks and specify the lanugage. It is important to finish a code block with the same number of backtics.

\begin{example}
\begin{lstlisting}
```html
<!DOCTYPE html>
<html lang="en">
<head>
    <meta charset="utf-8"/>
    <meta content="IE=edge" http-equiv="X-UA-Compatible"/>
    <meta content="width=device-width, initial-scale=1.0" name="viewport"/>
...

```

```ocaml
let x = 2 in
2+2
```

```python
def fun x:
  if x then x
  else x
```
\end{lstlisting}
\end{example} 

\subsection{Typesetting Mathematics}

You can use the usual LaTeX delimiters to include mathematics, i.e., for inline style, use single dollar sign.
%
For math blocks use double-dollars or \lstinline`\[`.

\subsection{Images}

\begin{flex}
\begin{gram}
You can include JPEG, PNG, and SVG images by using following syntax.
\begin{lstlisting}
![caption](image_file.ext){width=xy%}
\end{lstlisting}
\end{gram}

\begin{example}
The following code displays the image \lstinline`papers.svg`.
\begin{lstlisting}
![A categorization of CS papers.](papers.svg){width=90%}
\end{lstlisting}
\end{example}
\end{flex}

\subsection{Limitations}

Markdown to XML translation is relatively basic at this time.  If you need a particular feature not supported here, please let us know.
%
Here are few known limitations.
\begin{itemize}
\item Markdown usually allows blocks to be ended with equal or greater number of delimiter symbols than the start. For example a codeblock that starts with \lstinline!```! can be ended with \lstinline!```````!.  DC insists on blocks to start and end with the same length delimeters.

\item Blank lines in DC will always start an atom.  For example if you pepper blank lines into your list, each list item will turn into a separate atom.  This is probably not what you have intended.  
\begin{lstlisting}

* Item 1 is its own atom.

* Item 2 is its own atom.

\end{lstlisting}

\item DC does not allow nested paragraphs as some Markdown extensions do.  

\begin{lstlisting}

* Item 1 is its own atom.

    This is a paragraph that you intended to nest inside Item 1 but  this will start a new atom in DC.  

* Item 2 is its own atom.

\end{lstlisting}

\end{itemize}


\subsection{Compiling Markdown To XML}

The guide includes a very simple markdown document\lstinline`markdown/example.md`.

\begin{gram}[Making XML]
To generate xml, simply use the Makefile provided as follows.
%
\begin{lstlisting}
$ make markdown/example.xml
\end{lstlisting}
%
If you would like to know more about how thing are being handled, you can turn on the verbose option
%
\begin{lstlisting}
$ make verbose=true markdown/example.xml 
\end{lstlisting}
\end{gram}

\begin{gram}[Using DC directly]
You could also use the DC tools directly as follows.
%
The command will generate  \lstinline`markdown/example.xml`.

\begin{lstlisting}
$ dc markdown/example.md
\end{lstlisting}

\end{gram}

\begin{gram}[The ``Debugger''] 
As you might notice, \lstinline`dc` doesn't currenty give reasonable error messages, partly because if you can generate PDF and html from your LaTeX sources, it is unlikely for there to be errors. 
%
In case of an error, you can pass the \lstinline`-d` flag to \lstinline`dc` to tell her to print the atoms that it matches.
%
%See also \href{sec:dc::internals}{Section Internals} for additional debugging help.
\begin{lstlisting}
dc -d input.md -o output.xml
\end{lstlisting}
\end{gram}

%% \section{Internals}
%% \label{sec:dc::internals}

%% When using DC, a high-level understanding of the DC's internal could help.
%% %
%% DC works by splitting the LaTeX or Markdown document into atoms, which it treats as units.
%% %
%% It uses a Latex to HTML/XML or Latex to HTML/XML compiler to translate each atom and stitches the outputs back to construct the XML for the document.
%% %
%% The current implementation works with \href{https://www.pandoc.org}{pandoc}, though other compilers such as \href{https://dlmf.nist.gov/LaTeXML/}{LaTeXML} were also used at some point in the past.

%% When you receive an error, it will likely look like this:
%% \begin{lstlisting}
%% *latex_file_to_html: Executing command: pandoc --mathjax --lua-filter ./meta/codeblock.lua --lua-filter ./meta/span.lua /tmp/67.tex -o /tmp/67.html
%% Error in html translation.
%% Command exited with code: 65
%% Now exiting.make: *** [dc/dc.xml] Error 65
%% \end{lstlisting} 
%% % 
%% The error occurred when compiling atom \# 67, which is stored in the file \lstinline`/tmp/67.tex`.
%% %
%% To understand the error you can open this file.  The file is a complete LaTeX file and you can edit it and run \lstinline`pandoc` on it, possible with the flag \lstinline`--verbose`, which will likely be helpful.
%% %
%% You can ask \lstinline`pandoc` to generate a standalone html, by passing the \lstinline`-s` flag, so that you can open the result and examine how it looks rrin your browser.
%% %
%% \begin{lstlisting}
%% pandoc --verbose --mathjax --lua-filter ./meta/codeblock.lua --lua-filter ./meta/span.lua /tmp/67.tex -s -o /tmp/67.html
%% \end{lstlisting}
%% %
%% Once you figure out what the problem is what remains is to update your main document and proceed.
%% %
%% Basically the same workflow applies to Markdown inputs.


%% This is all you need to start withe.  Enjoy Diderot!
