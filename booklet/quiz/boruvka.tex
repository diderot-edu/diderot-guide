%\section{(MST) MST and Tree Contraction}

In \emph{SegmentLab}, you implemented Bor\r{u}vka's algorithm that
interleaved star contractions and finding minimum weight edges.  In
this question you will analyze Bor\r{u}vka's algorithm more carefully.

We'll assume throughout this problem that the edges are undirected,
and each edge is labeled with a unique identifier ($\ell$). The
weights of the edges do not need to be unique, and $m = |E|$ and $n =
|V|$.

\begin{lstlisting}[language=Caml, numbers=none]
(* returns the set of edges in the minimum  spanning tree of G *)

MST (G = (V,E)) = 
  if |E| = 0 then {}
  else let
    F = {min weight edge incident on v : v in V}
    (V',P) = contract each tree in the forest (V,F) to a single vertex
                  V' = remaining vertices
                  P = mapping from each v in V to its representative in V'
    E' = { (P_u, P_v, l) : (u, v,  l) in E  |  P_u <> P_v }
  in 
    MST (G' = (V',E')) union {l : (u,v,l) in F}
  end
\end{lstlisting}

\begin{problem}
\ask[4.]
Show an example graph with $4$ vertices in which \cd{F} will not include all
the edges of the MST.  Specify the graph as a set of vertices and edges, e.g., $V = \{0, 1, 2\}$ and $E = \{\{0,1\}, \{0,2\}\}$.

\sol
\begin{verbatim}
      3
   o --- o
 1 |     | 2
   o     o
\end{verbatim}

\ask[4.]
Prove that the set of edges $F$ must be a forest (i.e. $F$ has no cycle).

\sol
Answer 1: The MST does not have a cycle (it is a tree) and F is a subset of F
so it can't have a cycle.

Answer 2: AFSOC that there is a cycle.  Consider the maximum weight
edge on the cycle.  Neither of its endpoints will choose it since they
both have lighter edges.  Contradiction.
\end{problem}

\newpage

\begin{problem}[4.]
\ask
Suggest a technique to efficiently contract the forest in parallel.
What is a tight asymptotic bound for the work and span of your contract,
in terms of $n$? Explain briefly. Are these bounds worst case or
expected case?

\sol
Use star contraction as described in class. Since in contraction a tree
will always stay a tree, the number of edges must go down with the number
of vertices. Therefore total work will be $O(n)$ and span will be
$O(\log^2 n)$ in expectation.
\end{problem}

\begin{problem}[4.]
\ask
Argue that each recursive call to \sml{MST} removes, in the worst
case, at least \emph{half} of the vertices; that is, $|V'| \leq \frac{|V|}{2}$.

\sol
Every vertex will join at least one other vertex. Since edges have
two directions, at least $n/2$ of them must be selected, which will remove
at least $n/2$ vertices ($n = |V|$).
\end{problem}

\begin{problem}[4.]
\ask
What is the maximum number of edges that could remain after one step
(i.e. what is $|E'|$)? Explain briefly.

\sol
$m - n/2$ since at least $n/2$ edges are removed, as described in
previous answer.
\end{problem}

\begin{problem}[5.]
\ask
What is the expected work and span of the overall algorithm in terms of
$m$ and $n$? Explain briefly. You can assume that calculating $F$ takes
$O(m)$ work and $O(\log n)$ span.

\sol
Since vertices go down by at least a factor of $1/2$ on each round,
there will be at most $\log n$ rounds.  The cost of each round is
dominated by calculating $F$, $O(m)$ work and $O(\log n)$ span and the
contraction of forests $O(n)$ work and $O(\log^2 n)$ span.
Multiplying the max of each of these by $\log n$ gives $O(m \log n)$ work
and $O(\log^3 n)$ span.
\end{problem}

