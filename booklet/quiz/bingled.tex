%\section{(Sets and Tables) Bingled}

After forming your company Bingle to index the web allowing word
searches based on logical combination of terms (e.g. ``big'' and
``small''), you discover that there are already a couple companies out
there that do it....and lo-and-behold, they even have similar names.
You therefore decide to extend yours with additional features.  In
particular you want to support phrase queries: e.g. find all
documents where ``fun algorithms'' appears.

You decide the right way to represent the index is as a table of sets
where the keys of the table are strings (i.e. the words) and the
elements of the sets are pairs of values consisting of a document
identifier and an integer location in the document where the string
appears.  So, for example the following collection of three documents
with integer document identifiers:
%
\begin{quote}
\[
\begin{array}{ll}
\langle 
& (1, \cdm{``the~big~dog''}), 
\\
& (2, \cdm{``a~big~dog~ate~a~hat''}),
\\
& (3, \cdm{``i~read~a~big~book''})
\\
\rangle
\end{array}
\]
\end{quote}
~\\
the document index would be represented as
\[
\begin{array}{lll}
\cdm{idx} = & \{ & \cdm{``a''} \mapsto \cset{(2,0),(2,4),(3,2)}
\\
            &    &  \cdm{``big''} \mapsto \cset{(1,1),(2,1),(3,3)},
\\
            &   & \cdm{``dog''} \mapsto \cset{(1,2),(2,2)},
\\
            &   & \ldots
\\
            & \} &
\end{array}
\]

In particular you want to support the following interface
%
\begin{lstlisting}[language=Caml,numbers=none]
signature INDEX = sig
  type word = string
  type docId = int
  type index = docIdIntSet wordTable
  
  (* represents all documents and all locations where a phrase appears *)
  type docList

  val makeIndex : (docId * string) seq -> index    
  val find : index -> word -> docList
  val adj : docList * docList -> docList
  val toSeq : docList -> docId seq 
end
\end{lstlisting}
%
where, given an index \cd{idx},
%
\begin{lstlisting}[language=Caml,numbers=none]
toSeq (adj (find idx "210", find idx "rocks"))
\end{lstlisting}
%
would return a sequence of identifiers of documents
where ``210'' appears immediately before ``rocks'', and 
\\
\begin{lstlisting}[language=Caml,numbers=none]
toSeq (adj (find idx "Umut", adj (find idx "loves", find idx "climbing")))
\end{lstlisting}
~\\
would return a sequence of identifiers of documents
where the phrase ``Umut loves climbing'' appears.

\begin{problem}[8.]

\ask
Show the pseudocode for generating the index from the sequence of documents.
It should not be more than 8 lines of code and assuming all words have
length less than some constant, must run in $O(n \log n)$ work and
$O(\log^2 n)$ span, where $n$ is the total number of words across all
documents.    
%
You can use the function
%
\begin{lstlisting}[language=Caml,numbers=none]
toWords~:~string -> string seq
\end{lstlisting}
%
that breaks a text string into a sequence of words.

\solfin
\begin{lstlisting}[language=Caml, numbers=none]
type index = docIdIntSet wordTable

fun makeIndex (docs : (docId  * string) seq) : index =
  let
      ____ fun tagWords (id,doc) =                                     ____
      ____ let val words = toWords doc                                 ____
      ____ in                                                          ____
      ____ Seq.tabulate (fn i = (nth i words, (id, i)) (length words)  ____
      ____ end                                                         ____

      ____ val allPairs = Seq.flatten (Seq.map tagWords docs)          ____

      ____ val wordTable = Table.collect allPairs                      ____
      ____                                                             ____
      ____                                                             ____
      ____                                                             ____
      ____                                                             ____
in                  
    ____ Table.map Set.fromSeq wordTable                               ____
end
\end{lstlisting}
\end{problem}

\begin{problem}[8.]

Show code for a function \texttt{adj(docList1, docList2)} that given
two docLists returns a docList in which those two words
are adjacent.  For example for the index generated from the documents
above, 
\begin{lstlisting}[language=Caml,numbers=none]
toSeq(adj(find idx "a", find idx "big")) 
\end{lstlisting}
would return
$\cseq{2, 3}$.  For full credit \texttt{adj} must be an associative operator.

\ask
Define the \sml{docList} type and implement the function \sml{adj} as defined above.
You might find the function \sml{setmap} useful. The solution should
only be a few lines of code.
\begin{lstlisting}[numbers=none]
fun setmap f s = Set.fromSeq (Seq.map f (Set.toSeq s)) 
\end{lstlisting}


\solfin
\begin{lstlisting}[language=Caml,numbers=none]
type docList = (docIdIntSet) * int

fun adj (____      (d1,l1)      ____ , ____     (d2,l2)      ____ )  : docList = 
  let 
      ____ val d2' = setmap (fn (d,i) = (d, i-l1)) d2                     ____
  in  
      ____ (Set.intersection (d1,d2'), l1+l2)                             ____
  end
\end{lstlisting}


%% (* FYI: not part of exam *)
%% fun find idx word = 
%%   case Table.find idx word =>
%%     NONE => (Set.empty(), 1)
%%   | SOME d => (d, 1)
   
\end{problem}

