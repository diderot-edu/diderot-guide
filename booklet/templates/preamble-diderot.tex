\newcommand{\defn}[1]{\textbf{\emph{#1}}}
\newcommand{\ttt}[1]{\texttt{#1}}



% For html generation, we delegate some math to mathjax, 
% because pandoc doesn't deal with everything correctly.
\renewcommand{\htmlmath}[1]
{
\[
{#1}
\]
}

%%%%%%%%%%%%%%%%%%%%%%%%%%%%%%%%%%%%%%%%%%%%%%%%%%%%%%%%%%%%%%%%%%%%%%
%% BEGIN: PDFLATEX COMPATIBILITY
%% These make sure that we can use the same LaTeX sources to generate
%% PDF and XML
%% 
%% Do not modify this part. Skip to END: PDFLATEX...
%%%%%%%%%%%%%%%%%%%%%%%%%%%%%%%%%%%%%%%%%%%%%%%%%%%%%%%%%%%%%%%%%%%%%%

%% Diderot commands, download and attach are identity for html.
\renewcommand{\download}[2]{\download{#1}{#2}}
\renewcommand{\attach}[2]{\attach{#1}{#2}}

% We render \infer using MathJax fractions.
\newcommand{\infer}[2]{
\cfrac{#2}{#1}
}
%%%%%%%%%%%%%%%%%%%%%%%%%%%%%%%%%%%%%%%%%%%%%%%%%%%%%%%%%%%%%%%%%%%%%%
%% END: PDFLATEX COMPATIBILITY
%% These make sure that we can use the same LaTeX sources to generate
%% PDF and XML
%%%%%%%%%%%%%%%%%%%%%%%%%%%%%%%%%%%%%%%%%%%%%%%%%%%%%%%%%%%%%%%%%%%%%%
