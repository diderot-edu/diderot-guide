\chapter{Publishing Your Work On Diderot}
\label{ch:publish}

\begin{gram}
Diderot allows authors to generate interactive books from LaTeX and Markdown sources.
%
Interactive books, however, do raise a number of issues that are not straightforward to handle by starting with just LaTeX/Markdown sources, whose information content makes it difficult to enable certain operations such as live updates.
%
Another difficulty is the complexities of translating LaTeX/Markdown sources (especially LaTeX) to be online friendly.

%
Fortunately, Diderot offers an array of tools to deal with this complexity.
%
In this chapter, we first describe the basic concepts of publishing on
Diderot and then describe the structure of a Diderot ``book'' and a ``booklet'' and our recommendations for organizing these and creating content.
%
For brevity, we consider LaTeX sources here, but the practices generalize to Markdown sources.
\end{gram}

\begin{important}[Sources and Feedback]
To go through the examples in this chapter, please clone 
%
\href{https://github.com/diderot-edu/diderot-guide}{this github repository}
%
and follow the instructions in \lstinline`INSTALL.md`. 
%
The examples are tested on Mac OS X and Ubuntu.  The binaries in \lstinline`bin` might not work on systems that are not Mac or Linux/Unix-like. 

This guide is public and it source are available 
\href{https://github.com/diderot-edu/diderot-guide}{in this repo}.
%
Please send feeback by \href{https://github.com/diderot-edu/diderot-guide/issues}{opening an issue}.
\end{important}

\section{Released and Unreleased Content}
\label{sec:publish::released-unreleased}

You can publish content on Diderot in two different forms: ``unreleased'' and ``released''.
%
When you first upload some content in the form of a book chapter on Diderot, the chapter is unreleased by default.
%
Unreleased chapters are visible to course staff and they create various kinds of posts (e.g., feedback, questions, etc) on the content.
%
Unreleased chapters, however, are not visible to students.
%
You can release the chapter by using Book Managemement features. 
%
Doing so makes them visible to the students.
%
Before you release a chapter, pleate  beware that releasing should be treated as a one-way feature.
%

\begin{gram}[Released Content: Implications for Authors and Users]
\label{grm:publish::released::user-implications}
After you release a chapter, it becomes available to the students, who might start taking notes on its content and discussing it, by for example asking questions to course staff.
%
In other words, released content is out in the wild and has now affected students.
%

Updating content after it is released  therefore raises a number of questions, e.g.,
\begin{itemize}
\item should it be allowed at all? (For example, fixing an error pointed out by a student makes a good question look like an unnecessary question.)

\item if it should be allowed, then under what circumstances should it be allowed, who should decide?
\end{itemize}
\end{gram}

\begin{gram}[Technical Issues with Live Updates]
Beyond these important questions, live-updates  to released content are complex from a technical point of view.
%
The first issue is identifying what has been changed
and what has not.
%
To see why let's say that you have published and released a chapter on Diderot and later realized that you want to make some updates, and so you edited the chapter sources.
%
Now, if you want to update the released chapter on Diderot, then
Diderot needs to figure out which sections and atoms have been edited
and which ones remain the same.  
%
This is not easy, because Diderot doesn't know the full history of your edits.

%
A second technical concern is treatment of deleted content.  Let's say that you have deleted a paragraph, but a student took personal notes on this paragraph.  What should happen to students' notes?
%
Should they also be deleted? 
%
If not,  then they would be linked with deleted content, effectively making their notes unusable.  
\end{gram}

\begin{example}
Let's say you have edited a released chapter and swapped the position of two paragraphs.
%
This change is  ambiguous to Diderot, because it cannot
distinguish this edit from complete rewriting of the two paragraphs
(i.e., did you swap or did you rewrite?).
%
The distinction is important, because it impacts how the updates affect the students.
%
In the case of a swap, Diderot hardly needs to take any action.
%
In the case of a complete rewrite, Diderot would ideally allow the student to become aware of the change and take some action.
\end{example}


\begin{gram}[Releasing Content]
These challenges are not that hard to overcome but, at this time, we have decided to proceed with a simplifying principle: a chapter will not undergo major updates after it is released, but smaller updates, e.g., typo fixes, addition of some text, etc could be common. 
%
Diderot, however, does not allow deletion of an entire atom, e.g., a definition, a theorem, etc, though a deletion can be simulated by editing the content of that atom.
\end{gram}

\begin{gram}[Using  Atomic Files for Live Updates]
\label{grm:publisg::diderot-atomic}
When you use the Diderot compiler, \lstinline`dc`, in addition to an XML file, it will also generate an fully atomized or \defn{atomic} file.  For example, if you translate your source file \lstinline`main.tex` to XML by running

\lstinline`$ make main.xml`

the compiler will generate \lstinline`main.xml` and also the file 

\lstinline`diderot_main.tex`

This file is ``atomic'' in the sense that the contents has been partitioned into  blocks, representative units of Diderot.
%
Other than explicit marking of atom bounderies, the atomic file has the same content as the original \lstinline`main.tex`
%
and can be used to perform live updates. 
%
In particular you can edit the atomic file as you prefer as long as you don't change any of the existing labels and don't delete atoms or sections.  You can add new sections and atoms as long as you assign them labels.
 
\end{gram}


\begin{gram}[Atomic Sources]
%
The atomic file that the Diderot compiler generates is an instance of a ``fully labeled source file''.
%
Diderot makes the decision on whether a live update should proceed or not by an identification system that it uses to give a unique label to each block.
%
The authors can fully control this identification system, and thus also fully control live updates.
%
But doing so requires a bit more work from the authors: they need to given each block a unique label.
%
By generating automatically an atomic version of the source Diderot compiler \lstinline`dc` simplifies this task and thus live updates.
%

Please read 
%
\href{sec:publish::full-labeling}{the next section}
%
to learn more about this approach and a Diderot tool that can help generate labels.
\end{gram}


\begin{gram}[Best Practices]
The best practice for publishing content on Diderot is to publish it 
\begin{itemize}
\item in unreleased form first, 
\item inspect the material,
\item ideally, have your fellow co-instructor and TAs give you feedback (Diderot allows sending feedback on unreleased content), and 
\item update the content according to feedback, and release it.
\item Immediately after the release, save the LaTeX file (and its \lstinline`dc`-generated atomic version if desired) for live updates and error corrections.
\end{itemize}

This process aims to facilitate easy updates for unreleased content and reasonably easy updates for released content. 
\end{gram}

\begin{gram}[Live Update]
There are several ways to update ``live'', i.e., released, content.
%
If your sources are
\href{sec:publish::full-labeling}{fully labeled}, then live updates are easy.
%
If your sources are not fully labeled, then you can use your
%
\href{grm:publisg::diderot-atomic}{Diderot atomic file}
%
generated by the compiler to make updates.
%
In the unlikely even that you have lost that file, you may find a copy of it in the temporary directory supplied to the compiler.  The file will look like this \lstinline`Diderot_main_UUID.tex` where the string \lstinline`UUID` is replaced by a UUID string and \lstinline`main` is the name of the original file.
%

In the very unlikely event nearly all has been lost, there are still several ways to make live updates.

\begin{itemize}
\item \textbf{Regenarate the Atomic File:} Atomic file generation is
  deterministic.  That is the same content will generate the same
  atomic file.  Thus, if you have the original LaTeX source that you wish to
  update, then you can run the Diderot compiler again to regenerate the atomic file.
  After you have the atomic file, you may proceed to make your changes
  as you prefer see
%
 \href{grm:publisg::diderot-atomic}{Diderot atomic  file} 
%
for more details.
%

\item \textbf{Unreleasing:}
In some circumstances, you can simply \defn{unrelease} a chapter, effectively pulling it from user's view (excepting your course staff).
%
For example,
if you accidentally released a chapter and a small amount of time has passed, you might choose to unrelease a chapter so that you can make some updates.
%
Or if you released a chapter, and noticed that it is corrupt in a big way, you might choose to unrelease the chapter to prevent confusion among the users.
%

Once unreleased, you can update the chapters as you wish, and all other user interactions  prior to the updates will be deleted.
%
After you make your updates, you can release the chapter as usual.
%
Because unreleasing deletes all student interactions, this option should be used with care.

\item \textbf{Online edits:}
for small corrections, perhaps the easiest way is to edit the affected atoms online on Diderot.  For example, this is usually sufficient to fix small typos.  (Don't forget to update your LaTeX or Markdown source).

\item \textbf{XML edits:}
%
For larger changes, you can edit the XML of the chapter directly and
reupload the XML.  When editing the XML, it is very important not to
alter \lstinline`label` fields and update only the \lstinline`body` and \lstinline`title` fields.

\item \textbf{LaTeX edits:}
Certain LaTeX and Markdown edits are also feasible. 
%
For example, you can simply add new content to the tail of your LaTeX/Markdown chapter source but as you do this, it is very important to not alter any of the prior content. 
%
Similarly, if you want to change your preamble file (where your macros are defined), then you will be able to use your LaTeX sources to generate and XML and re-upload the file. 

\item \textbf{Versioning:} If none of the above work, you can ``abandon'' your released chapter.  Let's say that you released Chapter 2 a while back and noticed later some serious problems with it. You don't feel comfortable unreleasing it because it has been a week since its release, and the above solutions don't apply.  In this case, you can deprecate this chapter by changing its title to ``Deprecated'' and creating a new Chapter 2.1 (Chapters can be a real number with a single decimal, where the decimal is thought as a version number) with the updated content.  When uploading the new content, please remember to change the labels of the new chapter to make sure that it doesn't overlap with the old.  For example, if you have prefixed all your labels with 
%
\lstinline`chap2`
%
then replacing that string with 
%
\lstinline`chap2fixed:` 
%
could work.
\end{itemize}
\end{gram}

\begin{gram}[The Automatic Labeling Algorithm of Diderot]
It might help to understand the innerworkings of Diderot a bit to understand the limitations on live updates.
%
When you use Diderot's compiler to compile your sources to XML, the compiler generates a unique \defn{label} for each block (atom, group, or section).  
%
If the author has given a label to the block, then Diderot uses that label.
%
Otherwise, it generates one.
%
At a very high level, the labels are generated by using the text itself.  For example, if you have a definition that defines a ``graph'', Diderot's compiler might label it as \lstinline`def:graph-theory::graph`, where  
%
\lstinline`graph-theory` is the label of the chapter,
%
and the word \lstinline`graph` is pulled from the text of your definition.
%

Diderot uses labels as identifiers of all content.
%
For example, if a student asks a question about a definition atom, a link between the student and the definition (as identified by its label) is established.  
%
Likewise, if you update a released chapter, Diderot matches blocks based on their labels and updates the matching blocks in place.

Now, changing your latex sources could change the labels auto-generated by Diderot's compiler.  This in turn will lead to the upload deleting a block that is unmatched.
%
Therefore, simply giving a label to each of your atoms allows you to perform updates on released content.
%
This is the basic idea \href{sec:publish::full-labeling}{behind the full-labeling method}.
\end{gram}

\section{Full Labeling in LaTeX}
\label{sec:publish::full-labeling}

\begin{gram}
If you use LaTeX, then there is a way to write your sources to unlock live updates and enable ``continuous publishing''.
%
The idea is to atomize your chapter fully by making sure that each piece of text is inside of an atom and giving each section, group, and atom a unique label.
%
The labels given should not be changed between live updates, but could be changed otherwise (e.g., from one semester to the next).
%
\end{gram}


\begin{example}[An Fully Labeled Assignment]

Here is a fully labeled, or \defn{atomic}, assignment.  Note that the chapter, all sections, and all atoms have labels.  When assigning labels, note that they must be unique within the book; it is therefore good practice to prefix them with chapter label.
%

\begin{lstlisting}
\chapter{Assignment 1}
\label{ch:asg1}

\begin{preamble}
\label{prml:asg1::intro}
In this assignment, you will use your knowledge about complexity classes..
\end{preamble}

\section{The Class of P}
\label{sec:asg1::p}

\begin{problem}[String Matching in P]
\label{prb:asg1::p:string-matching}
Prove that an algorithm that checks that two string are equivalent are in P.
\end{problem}


\section{The Class of NP}
\label{sec:asg1::np}

\begin{problem}[String Matching in NP]
\label{prb:asg1::np:string-matching}
Prove that an algorithm that checks that two string are equivalent are in NP.
\end{problem}


\section{P versus NP}
\label{sec:asg1::p-versus-np}

\begin{problem}[P versus NP]
\label{prb:asg1::p-vs-np::p-vs-np}
Prove that if P = NP, then life is good.
\end{problem}

\end{lstlisting}
\end{example}


\begin{important}
The full labeling approach is usually feasible for small texts such as assignments but tends to be too labor intensive for larger text such as book chapters.
%
The Diderot compiler \lstinline`dc` can  automatically label chapters for you.  Simply run the compiler to generate the xml and it will give generate an atomic version your source for you, see
 \href{grm:publisg::diderot-atomic}{Diderot atomic  file} 
%
for more details.
\end{important}

Labeling all blocks in a chapter allows Diderot to match them when you want to update a released section: blocks with the same label are matched and the online content is updated with the new one.
%
This means that 
\begin{itemize}

\item you can edit the text within the atoms as you wish,
%

\item you can add new labeled blocks as long as you don't change the labels of existing ones, e.g., you can add new problems

\item you can ``delete'' content by leaving the atom in and delete its text.
\end{itemize}
%
If you make such updates, especially that changes or removes existing content, 
please be mindful of their 
%
\href{grm:publish::released::user-implications}{implications of live updates for the users}.

\begin{note}
You can \href{sec:dc::labels-refs}{read more about labels and references here.}
\end{note}

\section{Structuring your Sources}
\label{sec:publish::latex-structure}
%
We recommend following the structure
of the book and the booklet examples included in this guide source
code (in folders `book` and `booklet`) and described here.
%
As examples, 
the folders \lstinline`book` and \lstinline`booklet` in guide sources contain several files.
%
\begin{itemize}
\item \lstinline`book.tex`

This is the top level file for PDF generation.  It is a conventional LaTeX document is but is organized carefully for Diderot compatibility.

\item \lstinline`book-html.tex`

This is the top level file for html generation.  It is mostly a copy of \lstinline`book.tex` but it uses slightly different packages.

\item \lstinline`templates/diderot.sty`

Supplies diderot definitions needed for compiling latex to pdf's.
You don't need to modify this file.

\item \lstinline`templates/packages.sty`

Supplies packages for PDF generation.  You can add whatever package
you want.  Most of these packages, except perhaps basics ones and AMS
Math packages, will be ignored for XML generation.  It is important not to include any command definitions in this file.

\item \lstinline`templates/preamble.tex` 

Supplies your macros that will be used by generating a PDF via pdflatex.  All macros should be included here and no packages should be included (use \lstinline`packages.sty` for packages.  


\item \lstinline`templates/preamble-diderot.tex` 

Equivalent of \lstinline`preamble.tex` but it is customized for XML output.  This usually means that most macros will remain the same but some will be simplified to work with `pandoc`.  If you don't need to extensive customization, you can keep just one \lstinline`preamble.tex` and \lstinline`input` that file in your \lstinline`preamble-diderot.tex`, as in the \lstinline`booklet` example.
%
\begin{lstlisting}
% File: preamble-diderot.tex

% Includes all of preamble.tex
\newcommand{\defn}[1]{\textbf{\emph{#1}}}
\newcommand{\ttt}[1]{\texttt{#1}}


% Redefine \mybold command defined in preamble.tex for Diderot.
\renewcommand{\mybold}[1]{\textbf{#1}}
\end{lstlisting}
\end{itemize}    

\begin{important}
The idea behind organizing \lstinline`preamble-diderot.tex` by including \lstinline`preamble.tex` and modifying it as needed is to minimize human errors that can be more difficult to fix once published on Diderot. 
%
We strongly recommend the authors to follow this practice.
\end{important}



To simplify publishing on Diderot, we recommend organizing your booklet and book sources as outlined below. 

\begin{gram}[Booklets]
Booklets are books that don't have parts. We recommend creating one directory per chapter and placing a single \lstinline`main.tex` (a chapter specific name could also be used) file to include all contain that you want.  
%
Place all media (images, videos etc) under a media/ subdirectory. 
\begin{itemize}  
\item \lstinline`ch1/main.tex` or \lstinline`ch1/ch1.tex`
\item \lstinline`ch1/media/`: all my media files, *.png *.jgp, *.graffle, etc.
\item \lstinline`ch2/main.tex` or \lstinline`ch2/ch2.tex`
\item \lstinline`ch2/media/`: all my media files for chapter 2, *.png *.jgp, *.graffle, etc.
\item \lstinline`ch3/main.tex` or \lstinline`ch3/ch3.tex`
\end{itemize}
\end{gram}

\begin{gram}[Books]
Books have parts and chapters. We recommend structuring these as follows, where `ch1, ch2` etc can be replaced with names of your choice.
%
\begin{itemize}
\item \lstinline`part1/ch1.tex`
\item \lstinline`part1/ch2.tex`
\item \lstinline`part1/media-ch1/`
\item \lstinline`part1/media-ch2/`
\item \lstinline`part2/ch3.tex`
\item \lstinline`part2/ch4.tex`
\item \lstinline`part2/ch5.tex`
\item \lstinline`part2/media-ch3/`
\item \lstinline`part2/media-ch4/`
\item \lstinline`part2/media-ch5/`
\end{itemize}
   
\end{gram}


\begin{gram}[Making PDF and HTML]
Inside the \lstinline`book\` and \lstinline`booklet\` directories, 
you can use \lstinline`pdflatex` to generate a PDF output.  See the corresponding \lstinline`Makefile`.
%
For example, you can  invoke the \lstinline`Makefile` as follows to make a PDF:
\begin{lstlisting}
$ make pdf
$ make html
\end{lstlisting}
\end{gram}


\begin{gram}[Making PDF/HTML for a Single Chapter]
To make specific chapters,  extend the Makefile to compile each chapter separately.  See the \lstinline`Makefile` in book or booklet as examples.
%
For example, you can compile the chapter \lstinline`probability` in the \lstinline`book` directory as follows.
\begin{lstlisting}
$ make probability
\end{lstlisting}

To generate HTML for a single chapter, update \lstinline`book-html.tex` to exclude all other chapters and run \lstinline`make html`.

\end{gram}


\begin{gram}[Making XML]
XML generation proceeds on a chapter by chapter basis.
%
To generate xml, simply use the Makefile provided as follows.
%
\begin{lstlisting}
$ make ch2/main.xml
\end{lstlisting}
%

To obtain more information from a run, you can turn on the verbose option.
%
\begin{lstlisting}
$ make verbose=true ch2/main.xml
\end{lstlisting}

Beware that you will likely see many warnings.  For this reason the utility of the verbose flag is not particularly high.
\end{gram}

\begin{important}[PDF before XML]
Before generating the XML for a chapter, please check first that you can generate PDF for the chapter.  This is because XML generation is relatively ``forgiving'' and, for example, simply omits undefined macros without complaining.  By generating PDF first, the author could avoid many errors that can be more time-consuming to fix after the material is published on Diderot.
\end{important}

\begin{note}
The ``PDF before XML''
\end{note}

\begin{gram}
Error messages from the XML translator are not particularly useful at this time.  But, if you are able to generate a PDF and HTML (using \lstinline`pandoc`), then you should be able to generate an XML. 
%
If you encounter a puzzling error try the "debug" version which will print the atoms as it goes through the input.
%
\begin{lstlisting}
$ make debug=true probability/theory.xml
\end{lstlisting}

\end{gram}

\begin{gram}[Using DC directly]
Assuming that you structure your book as suggested above, then you will mostly be using the Makefile but you could also use the DC tools directly. 
%
This tools translates the given input LaTeX file to xml.

\begin{lstlisting}
$ dc  -meta ./meta -preamble preamble.tex input.tex -o output.xml
\end{lstlisting}

To turn on the debug mode, simply pass the flag \lstinline`-d`.
\begin{lstlisting}
$ dc  -d -meta ./meta -preamble preamble.tex input.tex -o output.xml
\end{lstlisting}

The meta folder contains some files that may be used in the XML translation.  You can ignore this directory to start with and then start populating it based on your needs.  The main file that you might want to add are Kate highlighting specifications to be used for highlighting code.
\end{gram}


%% ## Tool: tex2tex
%% This tools reads in your LaTeX sources, parses them, and writes it back.  It drops comments and normalized the whitespace but should otherwise return back a LaTeX file that is essentially the same as the input file.   You should not need to use this binary, which is primarily used for testing during development.

%% Examples: 
%% {lstlisting}
%% $ bin/tex2tex ./graph-contraction/star.tex -o ./s.tex
%% $ diff ./graph-contraction/star.tex ./s.tex
%% {lstlisting}
%% ### Tool: texel
%% This tool "normalizes" your latex sources.  This means that it

%% * atomizes your code, wrapping each paragraph into a non-descript "gram" atom if it is not already wrapped.

%% * wraps each atom by a "group", if not already wrapped by one.

%% * gives each segment (section, subsection, subsubsection, paragraph, atom) of the input file a label and it wraps each atom into a "group" if it is not already in a group.  A group is one of "cluster" "flex" "mproblem" (multipart problem).  

%% Generated labels have the form 
%% {lstlisting}
%% kind_prefix:chapter_label:segment_label
%% {lstlisting}
%% Here kind_prefix could for exmaple be
%% * `sec`, for section, subsection, subsubsection, paragraph
%% * `xmpl`, `thm`, for an example or a theorem.

%% The chapter_label is extracted from the chapter label given.  For exmaple, if the label has any one of the form 
%% {lstlisting}
%% ch:star | chapter:star | ch_star | ch__star | ch:_star | chap:_star
%% {lstlisting}
%% chapter_label will be `star`.

%% The tool takes the label, split it at the delimiters [:_]+ and if the prefix starts with "ch" it take the rest of the label as the chapter label.
 
%% Some example full labels:
%% * xmpl:star:simpleexample
%% * thm:star:costbound

