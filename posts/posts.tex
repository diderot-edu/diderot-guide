\chapter{Communication Subsystem}
\label{ch:posts}

Diderot has a fully featured communication subsystem that supports Q \&
A, discussions, polls, regrade requests, social posts, etc.
%
This communication system alone rivals other systems out there but it
goes further by tightly integrating communications with the rest of
the system, e.g., to support contextual communication.

\begin{gram}[Contextual Communication]
In education context is everything.
%
For example, it is difficult and sometimes even impossible to ask a
good question without carefully laying out the context.
%
Diderot directly supports contextual communication.
%  
For example,
\begin{itemize}
\item when reading their notes, a user can click on an
``example'' and ask a question about that example (the context is the example),
%

\item when reading a PDF document, a user can click on a page and
  ask a question about that page (the context is the page),

\item in a test (quiz or exam), a user can click on a test problem, and ask their question about that problem (the context is the problem),

\item when viewing a lecture video, a user can click on a the video
  and ask their question about that video (the context is the
  video),

\item when reviewing their graded assignment, a user can click on a graded problem and ask a question about their grading (the context is the graded problem).  
\end{itemize}
%
In all of these cases, Diderot ties the communication with the
context.  For example, it either directly displays the context to the
users or makes it easy to reproduce with the click of a button.
\end{gram}

\begin{gram}[Benefits of Contextual Communications]
We have found that contextual communication
\begin{itemize}
\item increases the ease and quality of interactions by helping the user to formulate their questions and answers,
\item helps users by seeing questions and discussions directly in their context,
\item naturally reduces redundancy, preventing the same questions
  being asked multiple times (because the users can easily see the
  questions in context),
\item increases efficiency by reducing the work required to attend to
  a specific request such as a question,
\item increases efficiency and scale by reducing the \textbf{total
  work} done by all instructors by allowing division of work, e.g., by
  automatically assigning tasks to individuals based on context.
\end{itemize}
\end{gram}
