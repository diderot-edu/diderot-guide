\chapter{An Overview of Diderot}
\label{ch:overview}

\begin{gram}
\label{ch:overview::intro}
Diderot is a relatively large system with many features. This guide presents a rough guide to some of these features.  Diderot is rapidly evolving, especially due to very high demand for new features and services. This guide is almost always out of date.
\end{gram}

\begin{gram}
\label{ch:overview::features}
At a high level, Diderot combines various features of online interactive books,  social networks, and LMSs (Learning Management Systems).
% 
Specific features of Diderot include the following.

\begin{itemize}

\item  
\href{ch:dc}{Publication tools} 
%
for transforming traditional plain-text (LaTeX and Markdown) and PDF documents to interactive, online books. 

\item 
Public and private books: instructors can make an online book publicly
available or make them private, by giving access only to enrolled
students.

%% \item Community courses: a community server that allows anybody to register for a community course by supplying an instructor-specified email domain (e.g., \@...cmu.edu, \@...edu).
 
\item \href{ch:lms}{Course management tools} for managing student enrollment and grades.

\item An \href{ch:posts}{integrated communication subsystem} that support contextual Q
  \& A and discussions, and more generally several different forms of
  communication.
  
\item An \href{ch:quiz}{online exam and quiz system} with an integrated Q \& A and discussions.  

\item A cloud-based code submission system and autograder. Students
  can submit their code for grading on a cloud-based
  autograder. Instructors can choose effectively any amount of compute
  power, ranging from small machine instances to larger ones.

\item A \href{ch:cli}{command-line-interface} that can be used to
  perform many task from the command line, e.g., uploading content,
  submitting assignments.

\item
  Cloud-based implementation that seamlessly scales and trivially
  deploys at any institution without requiring any onsite IT services.
\end{itemize}
\end{gram}
