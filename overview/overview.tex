\chapter{An Overview of Diderot}
\label{ch:overview}

\begin{gram}
Diderot is a relatively large system with many features. This guide presents a rough guide to some of these features.  Diderot is rapidly evolving, especially due to very high demand for new features and services and this guide is almost always out of date.

At a high level, Diderot combines various features of online interactive books,  social networks, and LMSs (Learning Management Systems).
% 
Specific features of Diderot include the following.
\end{gram}

\begin{itemize}

\item  
\href{ch:dc}{Publication tools} 
%
for transforming traditional plain-text (LaTeX and Markdown) and PDF documents to interactive, online books. 

\item 
Public and private books: instructors can choose to make their materials publicly available to everybody or can restrict access to their students, also on a selective (book by book) basis.

\item Community courses: a community server that allows anybody to register for a community course by using an instructor-specified email domain (e.g., \@...cmu.edu, \@...edu).
 
\item \href{ch:lms}{Course management tools} for managing student enrollment and grades.

\item An integrated Q \& A and discussions system. 

\item An \href{ch:quiz}{online exam and quiz system} with an integrated Q \& A and discussions.  

\item A cloud-based code submission system and autograder that can put
  effectively any amount of compute power at the fingertips of the
  user.

\item A \href{ch:cli}{command-line-interface} that can be used to perform many course management task from the command line.

\item
  Cloud-based implementation that seamlessly scales and trivially
  deploys at any institution without requiring any onsite IT services.
\end{itemize}
