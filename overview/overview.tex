\chapter{An Overview of Diderot}
\label{ch:overview}

\begin{gram}
\label{ch:overview::intro}
Diderot is a feature-rich online education system. This rough guide
presents an overview of some of these features.  Diderot is rapidly evolving,
especially due to very high demand for new services. This guide is
almost always out of date.
\end{gram}

\begin{gram}
\label{ch:overview::features}
At a high level, Diderot brings together features of publication
sytems, online books, social networks, and LMSs (Learning Management
Systems).
%
\textbf{The key idea behind Diderot is integration: the goal is to
  create one platform that provides all educational services required
  to teach a university or school course.}
%
Our experience so far shows that integration is a winner.
\begin{itemize}
\item Users like integration, because, for example, they don't have to juggle multiple systems.

\item Users are more productive, because everything is in one place
  and work well together. Users don't have waste time trying to
  integrate different systems.

\item Users can create online interactive books from existing
  materials in PDF, LaTeX, and Markdown.  Users don't have learn and
  rely on proprietary (often inefficient) UIs and don't have to
  relinquish the IP rights of their creative work to external
  companies.
  
\item Integration creates many new capabilities that are otherwise
  difficult or impossible.  These new capabilities increase efficiency
  by automating complex educational tasks that currently require
  highly skilled human labor (e.g., teaching assistants, teachers and
  professors).
  
\end{itemize}
%

\end{gram}

\begin{gram}[Features]
%
Specific features of Diderot include the following.

\begin{itemize}

\item
Tools for transforming PDF document to interactive online books.

\item  
\href{ch:dc}{Publication tools} 
%
for transforming traditional plain-text (LaTeX and Markdown) documents
to richly interactive, online books.

%


\item 
Access control: instructors can make an online book publicly
available or make them private, by giving access only to enrolled
students.

%% \item Community courses: a community server that allows anybody to register for a community course by supplying an instructor-specified email domain (e.g., \@...cmu.edu, \@...edu).
 
\item \href{ch:lms}{Course management tools} for managing student enrollment and grades.

\item An \href{ch:posts}{integrated communication subsystem} that support contextual Q
  \& A and discussions, and more generally several different forms of
  communication.
  
\item An \href{ch:quiz}{online exam and quiz system} with an integrated Q \& A and discussions.  

\item A cloud-based code submission system and autograder. Students
  can submit their code for grading on a cloud-based
  autograder. Instructors can choose effectively any amount of compute
  power, ranging from small machine instances to larger ones.

\item
  Analytics tools that allow instructors to monitor user
  engagement.

\item A \href{ch:cli}{command-line-interface} that can be used to
  perform many task from the command line, e.g., uploading content,
  submitting assignments.

\item  
  Cloud-based implementation that seamlessly scales and trivially
  deploys at any institution without requiring any onsite IT services.
\end{itemize}
\end{gram}
