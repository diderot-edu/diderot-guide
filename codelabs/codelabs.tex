\chapter{CodeLabs: A Cloud-Based Code Autograder}
\label{ch:codelabs}

Codelabs is a cloud-based code autograder.  To set it up, you will
need a Docker file for your grading environment.  That is pretty much
all.  In terms of the Autograder setup, CodeLabs is compatible with
Autolab and you can use your Autolab autograders and Make files as is.

The features of Codelabs include
\begin{itemize}
\item  
\textbf{manual grading:} allowing course staff to review code and leave comments and point
score on them,

\item
\textbf{regrade requests:} allowing students to create regrade
requests and course staff to respond to them

\item
  \textbf{submission management:} reviewing, rerunning/resubmitting submissions

\item
  \textbf{grade management:} viewing, downloading grades, and exporting them to gradebook.
\end{itemize}
%
Please see the video above by Ria Pradeep of Carnegie Mellon University for a brief tutorial on Codelabs.

In addition to serving as a standalone auto-grader, codelabs, can also
be used to provide automatic feedback to the students.
%
In this ``chapter-grader'' mode, the CodeLab is coupled with a submittable
(quiz) chapter.
%
When a student makes a submission on the chapter, the (coupled) CodeLab is invoked with the answers of the student.

\section{Standalone Mode}
\label{sec:codelabs::standalone}

\begin{gram}[Lab Setup]
You can specify various properties for the lab such as the deadlines, and other properties.  These should be self explanatory.  A couple of things to note


\begin{itemize}
\item \textbf{Assignment date} specifies the date for when the lab is assigned.  Diderot sorts  labs based on their assignment date.
\item \textbf{Handout} file can be of any type, e.g., a pdf, a tar ball, a compressed tar ball, etc.
\end{itemize}

\end{gram}

\begin{gram}[Autograder output format]
\label{sec:codelabs::standalone::output}

In the standalone mode, the autograder is expected to some output ending with a JSON string mapping problem names (as defined in the Grading tab) to the scores for that student, e.g.,
  
\begin{lstlisting}
{
  "scores":
  {"Shortest path": 5.0, "Longest path": 7.0, "Widest path": 15.0}
}
\end{lstlisting}

If the scoreboard is turned on and you would like to display a list of
submissions based on their score, include in the autograder aboutput a \lstinline`scoreboard` field, e.g.,


\begin{lstlisting}
{
  "scores":
  {"Shortest path": 5.0, "Longest path": 7.0, "Widest path": 15.0},
  "scoreboard": 
  7.0  
}
\end{lstlisting}
  
\end{gram}


\section{Chapter-Grader Mode}
\label{sec:codelabs::chapter}

\begin{gram}[Autograder output format]
\label{sec:codelabs::chapter::output}
  
In the chapter-grade mode, the autograder is expected to generate an
output than ends with a JSON string defining the problem names and the
scores for that submission, e.g.,
  
\begin{lstlisting}
{
  "problems":
    {"Shortest path": 10.0,
      "Longest path": 15.0
    },
  "scores":
    {"Shortest path": 5.0,
      "Longest path": 7.0}
}
\end{lstlisting}

The \lstinline`"problems"` key maps each problem to its point value.
The \lstinline`"scores"` key maps each problem to the score for that submission.
The problem as defined will determine the columns of the gradetable.
\end{gram}
