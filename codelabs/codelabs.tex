\chapter{CodeLabs: A Cloud-Based Code Autograder}
\label{ch:codelabs}

Codelabs is a cloud-based code autograder.  To set it up, you will
need a Docker file for your grading environment.  That is pretty much
all.  In terms of the Autograder setup, CodeLabs is compatible with
Autolab and you can use your Autolab autograders and makefiles as is.

Codelabs operates in two modes: standalone and chapter-grader. 
%
In the standalone mode, students submit their assignments to a CodeLab
by using the CodeLabs interface or the Diderot CLI.
%
In the chapter-grader mode, the CodeLab is coupled with a submittable
(quiz) chapter.
%
When a student makes a submission on the chapter, the (coupled) CodeLab is invoked with the answers of the student.

\section{Standalone Mode}
\label{sec:codelabs::standalone}

\begin{gram}[Autograder output format]
\label{sec:codelabs::standalone::output}

In the standalone mode, the autograder is expected to some output ending with a JSON string mappin problem names (as defined in the Grading tab) to the scores for that student, e.g.,
  
\begin{lstlisting}
{
  "scores":
  {"Shortest path": 5.0, "Longest path": 7.0, "Widest path": 15.0}
}
\end{lstlisting}
  
\end{gram}


\section{Chapter-Grader Mode}
\label{sec:codelabs::chapter}

\begin{gram}[Autograder output format]
\label{sec:codelabs::chapter::output}
  
In the chapter-grade mode, the JSON string should include additional information, including the rank for the problem (1, 2, 3, ...) and the total points available.  For example

\begin{lstlisting}
{"scores":
    {"1: Shortest path (10 pts)": 5.0,
     "2: Longest path (20 pts)": 7.0,
     "3: Widest path (30 pts)": 15.0,
 }}
\end{lstlisting}
\end{gram}
