%%%%%%%%%%%%%%%%%%%%%%%%%%%%%%%%%%%%%%%%%%%%%%%%%%%%%%%%%%%%%%%%%%%%%%
%% BEGIN: Your Macros Here
%%%%%%%%%%%%%%%%%%%%%%%%%%%%%%%%%%%%%%%%%%%%%%%%%%%%%%%%%%%%%%%%%%%%%%

%%%%%%%%%%%%%%%%%%%%%%%%%%%%%%%%%%%%%%%%%%%%%%%%%%%%%%%%%%%%%%%%%%%%%%
%% SHORTHANDS
%%%%%%%%%%%%%%%%%%%%%%%%%%%%%%%%%%%%%%%%%%%%%%%%%%%%%%%%%%%%%%%%%%%%%%
\renewcommand{\a}{{\alpha}\xspace}
\newcommand{\la}{\leftarrow}
\newcommand{\ra}{\rightarrow}
\newcommand{\dra}{\Rightarrow}
\newcommand{\sucht}{~\mid~}
\newcommand{\floor}[1]{\left\lfloor {#1} \right\rfloor}
%\DeclareMathOperator{\sucht}{\mid}
\newcommand{\ninfty}{{-\infty}}
\newcommand{\inq}{\stackrel{?}{\in}}
\newcommand{\xmark}{\ding{55}}

%%%%%%%%%%%%%%%%%%%%%%%%%%%%%%%%%%%%%%%%%%%%%%%%%%%%%%%%%%%%%%%%%%%%%%
%% DEFINITIONS
%%%%%%%%%%%%%%%%%%%%%%%%%%%%%%%%%%%%%%%%%%%%%%%%%%%%%%%%%%%%%%%%%%%%%%

% pandoc no like ensuremath
\renewcommand{\ensuremath}[1]{#1}

\newcommand{\lang}{\textsf{PARS}\xspace}


%%%%%%%%%%%%%%%%%%%%%%%%%%%%%%%%%%
% Random stuff
%%%%%%%%%%%%%%%%%%%%%%%%%%%%%%%%%%


%%%% Umut: Preferring colors, below.
%% % definition...puts in index in bold face
%% \newcommand{\defn}[2][]{
%% \underline{\textbf{#2}}\ifthenelse{\equal{#1}{}}{\index{\bf #2}}{\index{\bf #1}}}

\newcommand{\defn}[2][]{
\textcolor{black}{{\textbf{\emph{#2}}}\ifthenelse{\equal{#1}{}}{\index{\textbf{#2}}}{\index{\textbf{#1}}}}}

%\newcommand{\defn}[1]{
%\textcolor{black}{{\textbf{\emph{#1}}}}
%}




% puts in index in regular face
\newcommand{\indx}[2][]{\underline{\textbf{#2}}\ifthenelse{\equal{#1}{}}{\index{#2}}{\index{#1}}}

% underline
\newcommand{\un}[1]{\underline{#1}}

% currently only in setstables, either use everywhere or drop
\newcommand{\adt}[1]{{\normalfont\textsf{#1}}}

\newcommand{\probName}[1]{\textsc{#1}}

% Name of ficticious company, e.g. bingle
%% \newcommand{\company}[1]{
%% \textsf{\textbf{#1}$^{\mbox{{\tiny \textregistered}}}$}
%% }

\newcommand{\company}[1]{\textsf{\textbf{#1}}}

%% remarks
\newcounter{remark}[section]
\newcommand{\myremark}[3]{
\refstepcounter{remark}
~\\
{\bf $[$ \scriptsize{#1}.{\theremark}:}{\textbf{\scriptsize{#3}}$]$}
\\
}
\newcommand{\uremark}[1]{\myremark{Umut}{U}{#1}}
%\newcommand{\ur}[1]{\uremark{#1}}
\newcommand{\ur}[1]{}

\newcommand{\cwork}[1]{\ensuremath{W\left({#1}\right)}}
\newcommand{\cspan}[1]{\ensuremath{S\left({#1}\right)}}
\newcommand{\cworkof}[2]{\ensuremath{W_{#1}\left({#2}\right)}}
\newcommand{\cspanof}[2]{\ensuremath{S_{#1}\left({#2}\right)}}



%%%%%%%%%%%%%%%%%%%%%%%%%%%%%%%%%%%%%%%%%%%%%%%%%%%%
% Probability/expectation operators.  The ones ending in x should be
% used if you want subscripts that go directly *below* the operator
% (in math mode); no x means the subscripts go below and to the right.
% NB: \P is remapped below for the complexity class P.
%%%%%%%%%%%%%%%%%%%%%%%%%%%%%%%%%%%%%%%%%%%%%%%%%%%%
\renewcommand{\Pr}{\mathbf{Pr}}
\newcommand{\Prx}{\mathop{\mathbf{Pr}}\/}
\newcommand{\E}{\mathbf{E}}
\newcommand{\Ex}{\mathop{\bf E\/}}
\newcommand{\Var}{\mathbf{Var}}
\newcommand{\Varx}{\mathop{\mathbf{Var}\/}}
\newcommand{\Cov}{\textbf{Cov}}
\newcommand{\Covx}{\mathop{\textbf{Cov}\/}}

%% to complicated for mathjax 
%\newcommand{\prob}[2][\mbox{}]{\ensuremath{\mathop{\text{\normalfont {\textbf{Pr}}}}_{#1}\ifthenelse{\equal{#2}{}}{}{\left[#2\right]}}}

%\newcommand{\prob}[2][\text{}]{\mathbb{P}_{#1}\ifthenelse{\equal{#2}{}}{}{\left[#2\right]}}}
\newcommand{\prob}[1]{\mathbf{P}\left[{#1}\right]}
\newcommand{\probf}{\mathbf{P}}
\newcommand{\probprim}[1]{\mathbf{P}\left[\{#1\}\right]}
\newcommand{\probover}[2]{\mathbf{P}_{#1}\left[{#2}\right]}

%% to complicated for mathjax
%\newcommand{\expct}[2][\mbox{}]{\ensuremath{\mathop{\text{\normalfont
%        \textbf{E}}}_{#1}}\ifthenelse{\equal{#2}{}}{}{\left[#2\right]}}
\newcommand{\expctf}{\mathbf{E}}
\newcommand{\expct}[1]{\mathbf{E}\left[{#1}\right]}

\newcommand{\given}{{\ensuremath{\:\mid\:}}}
\newcommand{\onef}[1]{\mathbb{I}\left\{#1\right\}}

\newcommand{\event}[2]{\{#1 =#2\}}

%%%%%%%%%%%%%%%%%%%%%%%%%%%%%%%%%%
% Asymptotics
%%%%%%%%%%%%%%%%%%%%%%%%%%%%%%%%%%
\newcommand{\bigoh}[1]{O\left({#1}\right)}
\newcommand{\bigo}[1]{#1}
\newcommand{\bigtheta}[1]{\Theta\left({#1}\right)}
\newcommand{\bigomega}[1]{\Omega\left({#1}\right)}


%%%%%%%%%%%%%%%%%%%%%%%%%%%%%%%%%%
% Mathematical Relations
%%%%%%%%%%%%%%%%%%%%%%%%%%%%%%%%%%
\newcommand{\equivs}[2]{
\ensuremath{#1 \equiv #2}
}


\newcommand{\bigrho}{\scalebox{2.0}{\mbox{$\rho$}}}
\newcommand{\bigiota}{\scalebox{2.0}{\mbox{$\iota$}}}




%%%%%%%%%%%%%%%%%%%%%%%%%%%%%%%%%%%%%%%%%%%%%%%%%%%%%%%%%%%%%%%%%%%%%%
%%%%%%%%%%%%%%%%%%%%%%%%%%%%%%%%%%%%%%%%%%%%%%%%%%%%%%%%%%%%%%%%%%%%%%
%% LISTING ENVIRONMENT (lstlisting)
%%%%%%%%%%%%%%%%%%%%%%%%%%%%%%%%%%%%%%%%%%%%%%%%%%%%%%%%%%%%%%%%%%%%%%
%%%%%%%%%%%%%%%%%%%%%%%%%%%%%%%%%%%%%%%%%%%%%%%%%%%%%%%%%%%%%%%%%%%%%%

\usepackage{listings}

%% \newdimen\zzsize
%% \zzsize=11pt
%% \newdimen\kwsize
%% \kwsize=11pt


%% \newcommand{\basicstyle}{\ttfamily}
%% %\newcommand{\basicstyle}{\fontsize{\zzsize}{1.1\zzsize}\ttfamily}
%% \newcommand{\keywordstyle}{\ttfamily\bf}
%% %\newcommand{\keywordstyle}{\fontsize{\kwsize}{1.1\kwsize}\ttfamily\bf}

%% \newlength{\zzlstwidth}
%% \settowidth{\zzlstwidth}{{\basicstyle~}}
%% \newcommand{\lcm}{}

%% \lstset{
%% %  aboveskip=-0.5 \baselineskip,
%% %  belowskip=-0.8 \baselineskip,
%%   xleftmargin=5.0ex,
%%   basewidth=\zzlstwidth,
%%   basicstyle=\basicstyle,
%%   columns=fullflexible,
%%   captionpos=b,
%%   numbers=left, numberstyle=\small, numbersep=8pt,
%%   language=Python,
%%   keywordstyle=\keywordstyle,
%%   keywords={signature,sig,structure,struct,fun,fn,case,type,datatype,let,fn,in,end,functor,alloc,if,then,else,while,with,and,start,do},
%%   commentstyle=\rmfamily\slshape,
%%   morecomment=[l]{\%},
%%   lineskip={1.5pt},
%%   columns=fullflexible,
%%   numbers=none,
%%   keepspaces=true,
%%   mathescape=true,
%%   escapeinside={@}{@},
%% % NOTE: need TWO sets of braces around each definition below!
%%   literate={requires}{{$\lcm\text{\keywordstyle \% requires}$}}6
%%            {returns}{{$\lcm\text{\keywordstyle \% returns}$}}6
%%            {=}{{$\lcm=$}}2 
%%            {(}{{$($}}2
%%            {)}{{$)$}}2 
%%            {**}{{$\lcm\times$}}2 
%% %           {||}{{$\lcm\Vert$}}2 
%%            {|}{{$|$}}2 
%%            {fn}{{$\lcm\boldsymbol\lambda\hspace{-1ex}$}}1 
%%            {==>}{{$\lcm\boldsymbol.\hspace{-1ex}$}}1 
%% %           {=>}{{$\lcm\boldsymbol.\hspace{-1ex}$}}1 
%% %           {=>}{{$\lcm\boldsymbol.\hspace{-1ex}$}}1 
%%            {->}{{$\lcm\rightarrow$}}2 
%%            {'a}{{$\alpha$}}1 
%%            {'b}{{$\beta$}}1 
%% }

%%%% pandoc no like
%% %% lstinline patch, to inser mbox in math mode.
%% \newcommand*{\SavedLstInline}{}
%% \LetLtxMacro\SavedLstInline\lstinline
%% \DeclareRobustCommand*{\lstinline}{%
%%   \ifmmode
%%     \let\SavedBGroup\bgroup
%%     \def\bgroup{%
%%       \let\bgroup\SavedBGroup
%%       \mbox\bgroup
%%     }%
%%   \fi
%%   \SavedLstInline
%% }
%% %% End of lstinline patch

%%% \cd does not always work when called from within lstlisting, because you
%%% cannot use lst commands once espaping intolatex from lst
%%% environment.  Use \cdm for that purpose

%% pandoc no like lstinline
%\newcommand{\cd}[1]{\lstinline!#1!}
\newcommand{\cd}[1]{\texttt{#1}}
\newcommand{\cdfun}[1]{\mathit{#1}}
\newcommand{\cdm}[1]{\ensuremath{\mbox{\normalfont \texttt{#1}}}}

%% 
\newcommand{\cignore}{\_}
\newcommand{\ccomment}[1]{\text{(*}~\text{#1}~\text{*)}}

%% Syntax for sequences/sets/tables

\newcommand{\sseq}[1]{\mathbb{S}_{#1}}

% sequence
\newcommand{\cseq}[1]{\left\langle\, #1 \,\right\rangle}
\newcommand{\cseqb}{\left\langle\right.} %\hspace*{-1ex}}
\newcommand{\cseqe}{\left.\right\rangle}%{\hspace*{-1ex}\left.\right\rangle}
\newcommand{\cseqbb}{\left\langle\right.}
\newcommand{\cseqee}{\left.\right\rangle}
\newcommand{\clseq}{\left\langle}
\newcommand{\crseq}{\right\rangle}
\newcommand{\cseqf}[2]{\left\langle\, #1 \;|\; #2 \,\right\rangle}

% index range
\newcommand{\cirange}[2]{[#1 \cdots #2]}

% length
\newcommand{\cseqlen}[1]{\lvert #1 \rvert}  %% length should be


\newcommand{\cset}[1]{\left\{ #1 \right\}}
\newcommand{\csetf}[2]{\left\{ #1 \;|\; #2 \right\}}

\newcommand{\csetb}{\left\{\right.\hspace*{-1ex}}
\newcommand{\csete}{\hspace*{-1ex}\left.\right\}}
\newcommand{\csetsize}[1]{|{#1}|}

% Change the name of the following, used as table find
\newcommand{\cget}[2]{#1[#2]}

%\newcommand{\symiterate}{\prod}
%% \newcommand{\symreduce}{\sum}
%% \newcommand{\symscan}{\int}
%% \newcommand{\symscani}{\oint}

%% \newcommand{\symiterate}{\fbox{\mbox{\cd{iter}}}~}
%% \newcommand{\symreduce}{\fbox{\mbox{\cd{red}}}~}
%% \newcommand{\symscan}{\fbox{\mbox{\cd{scan}}}~}
%% \newcommand{\symscani}{\fbox{\mbox{\cd{scani}}}~}

%\newcommand{\symiterate}{\overline{\prod}}
%\newcommand{\symreduce}{\overline{\sum}}
%\newcommand{\symscan}{\overline{\int}}
%\newcommand{\symscani}{\overline{\oint}}

\newcommand{\mksym}[1]{\mathcal{#1}}
%\newcommand{\mkseqsym}[1]{\overline{\mksym{#1}}}
%% overline not needed, common case
\newcommand{\mkseqsym}[1]{\mksym{#1}}
% can drop tilde by overloading
\newcommand{\mkseqset}[1]{\widetilde{\mksym{#1}}}

\newcommand{\symappend}{{\mathop{++}}}
\newcommand{\symiterate}{\mkseqsym{I}}
\newcommand{\symiterateprefixes}{\mkseqsym{IP}}
\newcommand{\symreduce}{\mkseqsym{R}}
\newcommand{\symscan}{\mkseqsym{S}}
\newcommand{\symscani}{\mkseqsym{SI}}


%\newcommand{\kwiterate}[4]{\symiterate_{#1}^{{#2}~{#3}}~{#4}}
%% \newcommand{\kwiterate}[4]{
%%   \stackrel{\tiny {#2}~{#3}}
%%            {\stackrel{\symiterate}{{\tiny {#1}}}}
%%   ~{#4}
%% }

\newcommand{\kwappend}[2]{{#1}~\symappend~{#2}}

% mapping versions
\newcommand{\kwiterate}[4]{\cdm{iterate}~{#2}~{#3}~(\cdm{map}~{#1}~{#4})}
\newcommand{\kwiteratep}[4]{\cdm{iteratePrefixes}~{#2}~{#3}~(\cdm{map}~{#1}~{#4})}

\newcommand{\kwreduce}[4]{\cdm{reduce}~{#2}~{#3}~(\cdm{map}~{#1}~{#4})}
\newcommand{\kwscan}[4]{\cdm{scan}~{#2}~{#3}(\cdm{map}~{#1}~{#4})}
\newcommand{\kwscani}[4]{\cdm{scanI}~{#2}~{#3}(\cdm{map}~{#1}~{#4})}

%  direct versions
\newcommand{\kwiterated}[3]{\cdm{iterate}~{#1}~{#2}~{#3}}
\newcommand{\kwiteratepd}[3]{\cdm{iteratePrefixes}~{#1}~{#2}~{#3}}
\newcommand{\kwreduced}[3]{\cdm{reduce}~{#1}~{#2}~~{#3}}
\newcommand{\kwscand}[3]{\cdm{scan}~{#1}~{#2}~{#3}}
\newcommand{\kwscanid}[3]{\cdm{scanI}~{#1}~{#2}~{#3}}

%% \newcommand{\kwreducesuma}[4]{\sum_{#1}~{#2}~{#3}\left({#4}\right)}
%% \newcommand{\kwreducerhoa}[4]{\bigrho_{#1}~{#2}~{#3}\left({#4}\right)}

%% \newcommand{\kwreducesumb}[4]{\sum_{#1}{#4}~.w\left({#2}~{#3}\right)}
%% \newcommand{\kwreducerhob}[4]{\bigrho_{#1}{#4}~.w\left(\;{#2}~{#3}\right)}

%% \newcommand{\kwreducesumc}[4]{\sum_{{#1}.\left({#2}~{#3}\right)}{#4})}
%% \newcommand{\kwreducerhoc}[4]{\bigrho_{{#1}.\left({#2}~{#3}\right)}{#4})}


%% \newcommand{\kwreducesumd}[4]{\sum_{{#2}~{#3}}~{#1}.{#4}}
%% \newcommand{\kwreducerhod}[4]{\bigrho_{{#2}~{#3}}~{#1}.{#4}}

%% \newcommand{\kwreducesume}[4]{\sum_{{#2}~{#3}}~{#4}:{#1}}
%% \newcommand{\kwreducerhoe}[4]{\bigrho_{{#2}~{#3}}~{#4}:{#1}}

%% \newcommand{\kwreducesumf}[4]{\sum_{{{#2}~{#3}}_{#1}}~{#4}}
%% \newcommand{\kwreducerhof}[4]{\bigrho_{{{#2}~{#3}}_{#1}}~{#4}}



%%%%%%%%%%%%%%%%%%%%%%%%%%%%%%%%%%%%%%%%%%%%%%%%%%%%%%%%%%%%%%%%%
% Macros for referencing components
%%%%%%%%%%%%%%%%%%%%%%%%%%%%%%%%%%%%%%%%%%%%%%%%%%%%%%%%%%%%%%%%%

\newcommand{\indentedwidth}{0.99\textwidth}

\newcommand{\partref}[1]{Part~\ref{part:#1}}
\newcommand{\chref}[1]{Chapter~\ref{ch:#1}}
\newcommand{\chreftwo}[2]{Chapters \ref{ch:#1} and~\ref{ch:#2}}
\newcommand{\chrefthree}[3]{Chapters \ref{ch:#1}, and~\ref{ch:#2}, and~\ref{ch:#3}}
\newcommand{\secref}[1]{Section~\ref{sec:#1}}
\newcommand{\subsecref}[1]{Subsection~\ref{subsec:#1}}
\newcommand{\secreftwo}[2]{Sections \ref{sec:#1} and~\ref{sec:#2}}
\newcommand{\secrefthree}[3]{Sections \ref{sec:#1},~\ref{sec:#2},~and~\ref{sec:#3}}
\newcommand{\appref}[1]{Appendix~\ref{app:#1}}
\newcommand{\lstref}[1]{Listing~\ref{lst:#1}}
\newcommand{\lstreftwo}[2]{Listings~\ref{lst:#1} and~\ref{lst:#2}}

\newcommand{\figref}[1]{Figure~\ref{fig:#1}}
\newcommand{\figreftwo}[2]{Figures~\ref{fig:#1} and~\ref{fig:#2}}
\newcommand{\figrefthree}[3]{Figures~\ref{fig:#1}, \ref{fig:#2} and~\ref{fig:#3}}
\newcommand{\figreffour}[4]{Figures~\ref{fig:#1},~\ref{fig:#2},~\ref{fig:#3}~and~\ref{fig:#4}}
\newcommand{\figpageref}[1]{page~\pageref{fig:#1}}
\newcommand{\tabref}[1]{Table~\ref{tab:#1}}
\newcommand{\tabreftwo}[2]{Tables~\ref{tab:#1} and~\ref{tab:#1}}

\newcommand{\stref}[1]{step~\ref{step:#1}}
\newcommand{\caseref}[1]{case~\ref{case:#1}}
\newcommand{\lineref}[1]{Line~\ref{line:#1}}
\newcommand{\linereftwo}[2]{Lines~\ref{line:#1} and~\ref{line:#2}}
\newcommand{\linerefthree}[3]{Lines~\ref{line:#1},~\ref{line:#2},~and~\ref{line:#3}}
\newcommand{\linerefrange}[2]{lines \ref{line:#1} through~\ref{line:#2}}
\newcommand{\thmref}[1]{Theorem~\ref{thm:#1}}
\newcommand{\thmreftwo}[2]{Theorems~\ref{thm:#1} and~\ref{thm:#2}}
\newcommand{\thmpageref}[1]{page~\pageref{thm:#1}}
\newcommand{\lemref}[1]{Lemma~\ref{lem:#1}}
\newcommand{\lemreftwo}[2]{Lemmas~\ref{lem:#1} and~\ref{lem:#2}}
\newcommand{\lemrefthree}[3]{Lemmas~\ref{lem:#1},~\ref{lem:#2},~and~\ref{lem:#3}}
\newcommand{\lempageref}[1]{page~\pageref{lem:#1}}
\newcommand{\adtref}[1]{ADT~\ref{adt:#1}}
\newcommand{\corref}[1]{Corollary~\ref{cor:#1}}
\newcommand{\costref}[1]{Cost Specification~\ref{cost:#1}}
\newcommand{\defref}[1]{Definition~\ref{def:#1}}
\newcommand{\defreftwo}[2]{Definitions~\ref{def:#1} and~\ref{def:#2}}
\newcommand{\defpageref}[1]{page~\pageref{def:#1}}
\renewcommand{\eqref}[1]{Equation~(\ref{eq:#1})}
\newcommand{\eqreftwo}[2]{Equations (\ref{eq:#1}) and~(\ref{eq:#2})}
\newcommand{\eqpageref}[1]{page~\pageref{eq:#1}}
\newcommand{\grref}[1]{\ref{gr:#1}}
\newcommand{\ineqref}[1]{Inequality~(\ref{ineq:#1})}
\newcommand{\ineqreftwo}[2]{Inequalities (\ref{ineq:#1}) and~(\ref{ineq:#2})}
\newcommand{\ineqpageref}[1]{page~\pageref{ineq:#1}}
\newcommand{\itemref}[1]{Item~\ref{item:#1}}
\newcommand{\itemreftwo}[2]{Item~\ref{item:#1} and~\ref{item:#2}}
\newcommand{\propref}[1]{Proposition~[\ref{prop:#1}]}
\newcommand{\synref}[1]{Syntax~\ref{syn:#1}}

\newcommand{\exref}[1]{Example~\ref{ex:#1}}
\newcommand{\exrref}[1]{Exercise~\ref{exr:#1}}
\newcommand{\probref}[1]{Problem~\ref{prob:#1}}
\newcommand{\algref}[1]{Algorithm~\ref{alg:#1}}
\newcommand{\algreftwo}[2]{Algorithms~\ref{alg:#1} and~\ref{alg:#2}}
\newcommand{\dsref}[1]{Data Structure~\ref{ds:#1}}
\newcommand{\dtref}[1]{Design Technique~\ref{dt:#1}}
\newcommand{\solref}[1]{Solution~\ref{sol:#1}}




%%%%%%%%%%%%%%%%%%%%%%%%%%%%%%%%%%
% Binary search trees
%%%%%%%%%%%%%%%%%%%%%%%%%%%%%%%%%%
\newcommand{\bst}{\cd{BST}}
\newcommand{\key}{\cd{key}}
\newcommand{\data}{\cd{value}}
\newcommand{\cnode}{\cd{Node}}
\newcommand{\cleaf}{\cd{Leaf}}
\newcommand{\cexpose}{\cd{expose}}

%%%%%%%%%%%%%%%%%%%%%%%%%%%%%%%%%%
% Graphs
%%%%%%%%%%%%%%%%%%%%%%%%%%%%%%%%%%

% The name of a vertex
\newcommand{\vname}[1]{\cd{#1}}


%%%%%%%%%%%%%%%%%%%%%%%%%%%%%%%%%%
% Models and costs...not sure these are necessary
%%%%%%%%%%%%%%%%%%%%%%%%%%%%%%%%%%

% complexity classes
\newcommand{\PCLASS}{\mathbf{P}}
\newcommand{\NPCLASS}{\mathbf{NP}}

%% variants of PRAM
\newcommand{\pEREW}{\normalfont\textsf{EREW}\xspace}
\newcommand{\pPRAM}{\normalfont\textsf{PRAM}\xspace}
\newcommand{\pCRCW}{\normalfont\textsf{CRCW}\xspace}

%%%
%% Probability
%%%

\newcommand{\pmf}[1]{\mathbf{P}_{#1}}


%%%%%%%%%%%%%%%%%%%%%%%%%%%%%%%%%%%%%%%%%%%%%%%%%%%%
% Most of the rest are macros used for pseudocode
%%%%%%%%%%%%%%%%%%%%%%%%%%%%%%%%%%%%%%%%%%%%%%%%%%%%

%% Comments, right justified
\newcommand{\cdc}[1]{\cd{(* {#1} *)}} 
\newcommand{\cdch}[1]{\mbox{(* {#1} *)}} 

% Language names
\newcommand{\pml}{SPARC\xspace}
\newcommand{\Pml}{\pml}
\newcommand{\PML}{\pml}
\newcommand{\lc}{lambda calculus}

% Some fonts
\newcommand{\ttt}[1]{\texttt{#1}}
\newcommand{\sml}[1]{\cd{#1}}
\newcommand{\smlp}[1]{\cd{#1}}
\newcommand{\ctext}[1]{\mbox{\rm\em #1}}
%\newcommand{\com}[1]{\mbox{\textbf{#1}}}
\newcommand{\com}[1]{\cd{#1}}
%Code





%%%%%%%%%%%%%%%%%%%%%%%%%%%%%%%%%%
% Types
%%%%%%%%%%%%%%%%%%%%%%%%%%%%%%%%%%
\newcommand{\bools}{\mathbb{B}}
\newcommand{\ints}{\mathbb{Z}}
\newcommand{\nats}{\mathbb{N}}
\newcommand{\reals}{\mathbb{R}}

\newcommand{\uuu}{{\ensuremath{\mathbb{U}}}} 
\newcommand{\vvv}{{\ensuremath{\mathbb{V}}}}
\newcommand{\ccc}{{\ensuremath{\mathbb{C}}}}
\newcommand{\kkk}{{\ensuremath{\mathbb{K}}}}
\newcommand{\sss}{{\ensuremath{\mathbb{S}}}}
\newcommand{\tttt}{{\ensuremath{\mathbb{T}}}}
\newcommand{\bbb}{{\ensuremath{\mathbb{B}}}}
% The following are redundant, should clean up


\newcommand{\R}{\ensuremath{\mathbb{R}}}
\newcommand{\N}{\ensuremath{\mathbb{N}}}
\newcommand{\Z}{\ensuremath{\mathbb{Z}}}
\newcommand{\F}{\ensuremath{\mathbb{F}}}
\newcommand{\Q}{\ensuremath{\mathbb{Q}}}
\newcommand{\C}{\ensuremath{\mathbb{C}}}


\newcommand{\tybool}{\bools}
\newcommand{\tyint}{\ints}
\newcommand{\tyreal}{\reals}
\newcommand{\tynat}{\nats}
\newcommand{\tyopt}[1]{\ensuremath{{#1}~\cd{option}}}
\newcommand{\tyseq}{Seq\xspace}

%% 
\newcommand{\cdvar}[1]{\mathit{#1}}
\newcommand{\cunit}{\cd{(\,)}}
\newcommand{\carray}{\cd{array}}
\newcommand{\ctunit}{\cd{unit}}
\newcommand{\ctrue}{\cd{true}}
\newcommand{\cfalse}{\cd{false}}
\newcommand{\cand}{\cd{and}}
\newcommand{\cor}{\cd{or}}
\newcommand{\candnot}{\cd{andnot}}

\newcommand{\cminus}{\cd{-}}
\newcommand{\cplus}{\cd{+}}

% Pseudocode
\newcommand{\cappend}{\com{++}}
\newcommand{\ccase}{\com{case}}
\newcommand{\cof}{\com{of}}
\newcommand{\cdra}{\com{=>}}
\newcommand{\clet}{\com{let}}
\newcommand{\cin}{\com{in}}
\newcommand{\cend}{\com{end}}
\newcommand{\cif}{\com{if}}
\newcommand{\cthen}{\com{then}}
\newcommand{\celse}{\com{else}}
\newcommand{\cpfor}{\com{parfor}}
\newcommand{\cval}{\com{val}}
\newcommand{\cfn}[2]{\cd{lambda}~{#1}\,.\,#2}
\newcommand{\cderef}[1]{\com{!}{#1}}
\newcommand{\cmodule}{\com{module}}

%\newcommand{\cfun}[3]{\com{fun}\; #1~#2 \Rightarrow #2}
% following three perhaps clean up
\newcommand{\cpar}{\mid\mid}
\newcommand{\ctype}{\com{type}\xspace}
\newcommand{\corder}{\com{order}}
\newcommand{\cless}{\com{less}}
\newcommand{\cequal}{\com{equal}}
\newcommand{\cgreater}{\com{greater}}

% Some shorthands for mathematical expressions
\newcommand{\pparen}[1]{\left(#1\right)}
\newcommand{\norm}[1]{\ensuremath{\left\Vert #1 \right\Vert}}
\newcommand{\vareps}{\varepsilon}
\newcommand{\fsqrt}[1]{\left\lfloor\sqrt{#1}\,\right\rfloor}
\newcommand{\csqrt}[1]{\left\lceil\sqrt{#1}\,\right\rceil}
\newcommand{\isqrt}[1]{\left\lceil\sqrt{#1}\,\right\rceil}

% Constants
\newcommand{\csome}[1]{\cd{Some}~{#1}}
\newcommand{\cnone}{\cd{None}}

% The next two are redundant
\newcommand{\mksome}[1]{~\mathit{Some(#1)}}
\newcommand{\mknone}{~\mathit{None}}


% Functions on sequences/sets/tables
\newcommand{\cscan}{\cd{scan}}
\newcommand{\cscani}{\cd{scanI}}
\newcommand{\creduce}{\cd{reduce}}
\newcommand{\ccollect}{\cd{collect}}
\newcommand{\cmap}{\cd{map}}
\newcommand{\citer}{\cd{iter}
}
%%% These don't work with pandoc
%\DeclareMathOperator*{\argmin}{arg\,min}
\DeclareMathOperator*{\argmax}{arg\,max}
% The following two are used for ordered sets
% Should change the name since confused with tuple access
\newcommand{\cpair}[2]{\left( {#1},~{#2} \right)}
\newcommand{\cfirst}{\cd{first}\xspace}
\newcommand{\clast}{\cd{last}\xspace}
\newcommand{\cinter}{\cd{intersection}}
\newcommand{\cdiff}{\cd{difference}}



%%%%%%%%%%%%%%%%%%%%%%%%%%%%%%%%%%
% Characters Strings
%%%%%%%%%%%%%%%%%%%%%%%%%%%%%%%%%%

%\newcommand{\str}[1]{\texttt{`#1'}}
\newcommand{\str}[1]{\text{'}\,{#1}\,\text{'}}
% Code string
\newcommand{\cstr}[1]{\str{\cd{#1}}}
% Math string, 
\newcommand{\mstr}[1]{\str{{#1}}}
\newcommand{\chr}[1]{\str{#1}}
\newcommand{\cchr}[1]{\cstr{#1}}


%%%%%%%%%%%%%%%%%%%%%%%%%%%%%%%%%%
% Sequences / Sets / Tables
%%%%%%%%%%%%%%%%%%%%%%%%%%%%%%%%%%


%%%%%%%%%%%%%%%%%%%%%%%%%%%%%%%%%%%%%%%%%%%%%%%%%%%%
% Denotational semantics
%%%%%%%%%%%%%%%%%%%%%%%%%%%%%%%%%%%%%%%%%%%%%%%%%%%%%%%%%%%%%%%%%%%%%%

\newcommand{\set}[1]{\left\{#1\right\}}
\newcommand{\setunion}{\cup}
\newcommand{\setint}{\cap}
\newcommand{\dom}[1]{\textsf{dom}(#1)}

\newcommand{\means}[1]{\left[\lvert {#1} \rvert\right]}
%\newcommand{\means}[1]{\left\llbracket {#1} \right\rrbracket}


%%%%%%%%%%%%%%%%%%%%%%%%%%%%%%%%%%%%%%%%%%%%%%%%%%%%
% Some stray definitions 
%%%%%%%%%%%%%%%%%%%%%%%%%%%%%%%%%%%%%%%%%%%%%%%%%%%%%%%%%%%%%%%%%%%%%%
\newcommand{\Rset}{\mathcal{T}}

%%%%%%%%%%%%%%%%%%%%%%%%%%%%%%%%%%%%%%%%%%%%%%%%%%%%
% Chapter specific definitions
%%%%%%%%%%%%%%%%%%%%%%%%%%%%%%%%%%%%%%%%%%%%%%%%%%%%%%%%%%%%%%%%%%%%%%
\newcommand{\eval}[1]{\mbox{Eval}(#1)}

%% BEGIN: Genome
\newcommand{\covlp}{\cdm{overlap}}  
\newcommand{\cjoin}{\cdm{join}} 
%% END: Genome

%% These are defined by amsmath
%\newcommand{\max}{\textsf{max}}
%\newcommand{\min}{\textsf{min}}
\newcommand{\paral}{\overline{P}}

%% BEGIN: Design chapter
\newcommand{\MCS}{\textsf{MCS}}
\newcommand{\MCSS}{\textsf{MCSS}}
\newcommand{\MCSSE}{\textsf{MCSSE}}
\newcommand{\MCSSS}{\textsf{MCSSS}}
\newcommand{\MCSSPS}{\textsf{MCSSPS}}
\newcommand{\mcsss}[2]{\mbox{\texttt{AlgoMCSSS}}(#1,#2)}

%% \newcommand{\MCS}{\textsf{MCS}\xspace}
%% \newcommand{\MCSS}{\textsf{MCS2}\xspace}
%% \newcommand{\MCSSE}{\textsf{MCS2E}\xspace}
%% \newcommand{\MCSSS}{\textsf{MCS3}\xspace}
%% \newcommand{\MCSSPS}{\textsf{MCS2PS}\xspace}
%% \newcommand{\mcsss}[2]{\mbox{\texttt{AlgoMCS3}}(#1,#2)}

%% BEGIN: randomized chapter
\newcommand{\qsort}{quick sort}
\newcommand{\Qsort}{Quick sort}

\newcommand{\SSpace}{\ensuremath{\Omega}}
\newcommand{\EventE}{\mathcal{E}}

\newcommand{\randomizedmaxtwostart}{10}
\newcommand{\randomizedmaxtwocmpone}{4}
\newcommand{\randomizedmaxtwocmptwo}{6}
\newcommand{\randomizedosfilterone}{4}
\newcommand{\randomizedosfiltertwo}{5}
\newcommand{\randomizedqsortpivot}{5}

\newcommand{\ksmall}{\cd{select}}

%% END: randomized chapter

%% BEGIN: bst chapter

\newcommand{\bstt}{\tttt}
% create a left paren of given height
\newcommand{\leftparen}[1]{\left(\\[#1]\right.}
% create a right paren of given height
\newcommand{\rightparen}[1]{\left.\\[#1]\right)}
%% END: bst chapter

%% BEGIN: Graphs
\newcommand{\vertu}{\mathcal{V}}

\newcommand{\alice}{\mbox{Alice}}
\newcommand{\arthur}{\mbox{Arthur}}
\newcommand{\bob}{\mbox{Bob}}
\newcommand{\josefa}{\mbox{Josefa}}
%% END: Graphs


%% BEGIN: Graph search
\newcommand{\bfs}{\ttt{BFS}}
\newcommand{\linegschoose}{6}
\newcommand{\linebfstreeflatten}{11}
\newcommand{\linebfstreeinject}{12}
\newcommand{\linebfstreenewfrontier}{13}
\newcommand{\finish}{finish}
\newcommand{\revisit}{revisit}
\newcommand{\visit}{visit}
\newcommand{\visittime}{\mathcal{T}_{\mathcal{V}}}   %% as in visit time
\newcommand{\finishtime}{\mathcal{T}_{\mathcal{F}}}   %% as in visit time
\newcommand{\ancestors}{\mathit{ancestors}}  
\newcommand{\flag}{\mathit{flag}}  
%% END: Graph search


%% BEGIN: Shortest Paths

\newcommand{\visited}{X}
\newcommand{\unvisited}{T}
\newcommand{\ev}{\sml{eVal}}
\newcommand{\dist}{\delta}
\newcommand{\sssp}{\probName{SSSP}}
\newcommand{\ssspp}{\probName{SSSP$^+$}}
\newcommand{\ssspd}{\probName{SSSP$_{\dist}$}}
\newcommand{\sssppd}{\probName{SSSP$^+_{\dist}$}}

\newcommand{\linedijkstramin}{7}
\newcommand{\linedijkstrafind}{10}
\newcommand{\linedijkstralet}{13}
\newcommand{\linedijkstrainsert}{14}
\newcommand{\linedijkstrapqinsert}{15}
\newcommand{\linedijkstraiter}{16}

\newcommand{\linebfdistances}{6}
\newcommand{\linebfnegcycle}{9}
\newcommand{\linebfif}{10}

%% END: Shortest Paths

%% BEGIN: Graph Contraction
\newcommand{\cheads}{\cd{heads}}
\newcommand{\nn}{\ensuremath{n_\bullet}}
\newcommand{\linegcstarmerge}{6}
\newcommand{\linegcstarself}{10}

\newcommand{\linegcscpartition}{6}
\newcommand{\linegcscedges}{7}
\newcommand{\linegcncback}{10}

\newcommand{\linegcconnectccpartition}{6}
\newcommand{\linegcconnectccedges}{7}


%% END: Graph Contraction

%% BEGIN: Graph Contraction
%% Does not work with pandoc.
%\newcommand{\boruvka}{Bor\r{u}vka\xspace}
\newcommand{\boruvka}{Boruvka\xspace}
\newcommand{\linemstminedges}{11}
\newcommand{\linemsthooks}{12}
\newcommand{\linemstnewvertices}{13}
%% END: Graph Contraction


%% BEGIN: Hashing
\newcommand{\kuni}{\mathcal{U}}
\newcommand{\kallhash}{\mathcal{H}}
\newcommand{\natspre}[1]{\nats_{<~{#1}}}
\newcommand{\kloadfac}{\alpha}
\newcommand{\nconf}{C}
\newcommand{\xconf}[1]{C_{{#1}}}
\newcommand{\xconfp}[2]{C_{{#1}, {#2}}}

%% END: Graph Contraction




%%%%%%%%%%%%%%%%%%%%%%%%%%%%%%%%%%%%%%%%%%%%%%%%%%%%%%%%%%%%%%%%%%%%%%
%% END: Your Macros
%%%%%%%%%%%%%%%%%%%%%%%%%%%%%%%%%%%%%%%%%%%%%%%%%%%%%%%%%%%%%%%%%%%%%%
