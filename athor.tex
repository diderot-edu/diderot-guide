\chapter{Diderot Guide}
\label{ch:guide}

%
% PREREQUISITE: 
% Create ``Slides'' book for the author. 

Let's start with creating and uploading chapters.  Let's use the
"Slides" book that I created (you can create as many books as you
want, see below).  Go to "Manage Books" from the "Links" on the center
right, to the left of search bar on top.  Now you can create a
chapter.  

\section{Chapters}
\label{guide:chapter}

\begin{gram}[Chapter creating]
\label{guide:chapter::create}
You can add a chapter to a book.  From the book's table of contents, press the ``settings'' icon next to its title.  This will take you to the book's management screen.  Here press ``Create Chapter.''
%
You can reach the same management screen from the "Manage Books" link
from the "Links" button on the center right, to the left of search bar
on top.  

When creating a chapter, assign the chapter a unique number and a unique label \ttt{ch:kleene}.  The label needs to be unique only within the book---no two chapters of the same book can have the same label.
%
If the chosen label does not meet this criteria, you will receive an error message and can choose another one.


As an example, create now a chapter for your second slide deck from
your Fall 2018 site and give it the label \ttt{ch:kleene}.
\end{gram}

\begin{gram}[Released versus Unreleased Chapters]
A chapter is either released or unreleased. 
%
\defn{Released} chapters are visible to the students.
%
\defn{Unreleased} chapters are not visible to the students but  are visible to course staff.  

This is designed to allow for a ``feedback cycle'' before releasing chapters to students.  
%
For example, I typically  upload my lectures a bit ahead of time and ask my TAs to look over them and give me feedback.
%
To give feedback, the TA's simply select the relevant atom and create a feedback, e.g., they might note a typo.
%
I then fix these problems, reupload the chapter, and then release it.  
\end{gram}

\begin{gram}[Release Dates and Schedule]  
You can assign release dates to your chapters and this will
automatically construct (and maintain) a schedule for you.  Let's skip
this step for now.  So leave these fields empty.  You can edit
chapters to add schedule information later.
\end{gram}


\begin{gram}[User Assignments]
You can "assign" users to a chapter.  The intention here is to
designate \defn{discussion chairs}, who are usually TAs for each
chapter. 
%
They will be assigned the questions asked on that content.

I don't know who I will assign the chapters to at the time of creation and therefore leave this blank.  I or the TAs later edit the chapter to assign the discussion chairs. 
\end{gram}

\begin{gram}[Uploading Content]
\label{guide:chapter::upload}
To upload contents into a chapter, move the mouse over the chapter and
select upload content.  Here you can upload either an XML
file or a PDF.  Let's start with PDF.  Select the PDF from
slides and select the PDF file to upload, and upload.  

Now you will see a spinning wheel.  This will take some time maybe a
minute or two.  What is going on is that the PDF is split into pages
and the text is extracted from it.  When it is finished, click on the
chapter and you should be able to view it.  

As an example, upload your second slide deck from your Fall 2018 site to the chapter that you have created above.
\end{gram}

\begin{important} 
When a PDF document is uploaded, its text is extracted from and stored into an index, allowing the document to be searchable via the search bar when the chapter is released.
%
Unreleased chapters will not be searchable (so students won't see
unreleased content).
\end{important}

\begin{note}
Currently, pdf documents themselves are shown as JPEG images.  We are in the process of changing this so that they are shows an PDFs.  This should be ready soon.
\end{note}


\begin{gram}[Actions on Chapters]
Once a chapter is uploaded, it becomes "actionable".  

\begin{itemize}
\item
If unreleased, course staff can send feedback and open other discussions. 
\item
If released: all the students could send feedback, ask questions, take private notes on the content.
\end{itemize}
\end{gram}


\begin{gram}[Updating a Chapter]

For unreleased chapters, simply re-upload the contents.  This will
recreate the chapter and delete all prior discussions on it.  

For released chapters, you can unrelease and re-upload, but this will
delete all content, including student created content (private notes,
questions, etc). 
%
Unreleasing and re-uploading could be thought as the same as deleting the chapter and recreating it.
%
Because it deletes all related content, we discourage using this
feature.

There are more advanced ways to upload released content. We will see
these later.
\end{gram}

\section{Books}
\label{guide:books}

\begin{gram}[Create a book]
You can create new books and as many as you want.  I usually create one for textbook, one for recitations, one for assignments, one for additional materials (misc).  Many courses create a "Slides" book.  

To create a book go to "links" right next to search bar and then press
manage books.  You will give your book a unique label (e.g,
"textbook", "recitations", "misc"...) and a unique"rank" which is the
order it will be shown but is otherwise immaterial.  


By default, a book is selected to be a \defn{booklet}, which
consists of a sequence of chapter.  
%
Iuf you want the book to contain "parts" as an organization element
above chapters, uncheck the ``booklet'' box.  

A book can contain PDFs or XMLs (generated by a compiler).
\end{gram}

\begin{example}
In the algorithms (15210) course, the main textbook has parts but
the rest of the books don't.  So all the books except for the textbook are booklets.
\end{example}


\begin{gram}[Editing a Book]
To edit the book, create and upload chapters, go to "Manage Books" and select your book.  Here you can edit various properties of the book, 
%
\href{create chapters}{guide:chapter::create}, 
%
\href{upload content}{guide:chapter::upload}, etc.
\end{gram}


\section{User Accounts} 

\begin{gram}[Creation]
To create user accounts, you can upload a roster in CSV format, which will email them all.  
%
I usually wait to do that until the first day of the class or right before it but you can create accounts for your TAs a bit ahead of time so that they can start playing with things (TAs and undergrads seem to pick things up super quickly).  When you create accounts, the users receive email notifications and instructions on how to log on.
\end{gram}

