\chapter{Diderot Guide}
\label{ch:guide}

Let's start with creating and uploading chapters.  Let's use the
"Slides" book that I created (you can create as many books as you
want, see below).  Go to "Manage Books" from the "Links" on the center
right, to the left of search bar on top.  Now you can create a
chapter.  Here is a direct link to 
%\href{manage %books}{https://www.diderot.one/courses/manage\_course/?course\_pk=12}

\section{Chapter Creation}

\begin{gram}[Label]

Give the chapter a number and a unique label (unique within the book), e.g., \ttt{ch:kleene}.  

\end{gram}

\begin{gram}[Released versus Unreleased]
A chapter is either \defn{released} and \defn{unreleased}. 
%
Released chapters are visible to the students and unreleased ones are not. Unreleased chapters are visible to course staff.  

How to use this? You can upload your lectures a bit ahead of time and ask for your TAs to look over them and give you feedback (on diderot).  You can then fix the problems, reupload, and then release them.  After you release a chapter, there are several ways to update.  I will explain these in a subsequent email.  As an example, create now a chapter for your second slide deck from your Fall 2018 site. 
\end{gram}

\begin{gram}[Release Dates and Schedule]  

You can assign release dates to your chapters and this will
automatically construct (and maintain) a schedule for you.  Let's skip
this step for now.  So leave these fields empty.  You can edit them
later.
\end{gram}


\begin{gram}[Assignments]
You can "assign" users to a chapter.  The intention here is to
designate \defn{discussion chairs}, who are usually TAs for each
chapter. 
%
They will be assigned the questions asked on that content.
Again, skip, can edit later.
\end{gram}

\begin{gram}[Uploading Content]
To upload contents into a chapter, move the mouse over the chapter and
select upload content.  Here you can upload either an XML
file or a PDF.  Let's start with PDF.  Select the PDF from
slides and select the PDF file to upload, and upload.  

Now you will see a spinning wheel.  This will take some time maybe a
minute or two.  What is going on is that the PDF is split into pages
and the text is extracted from it.  When it is finished, click on the
chapter and you should be able to view it.  

An important point: because the text is extracted from the slides, all
uploaded slides are searchable through the search bar above.  Note
however, that unreleased chapters will not be searchable (so students
won't see unreleased content).
\end{gram}

\begin{gram}[Usage]
Once you upload a chapter, it becomes "actionable".  If unreleased students can send feedback and open other discussions. Once released, all the students could do the same.  
\end{gram}

\begin{gram}[Updating a Chapter]

For unreleased chapters, simply re-upload the contents.  This will
recreate the chapter and delete all prior discussions on it.  

For released chapters, you can unrelease and re-upload, but this will
delete all content, including student created content (private notes,
questions, etc).  So I don't recommend this.  There are more advanced
ways to upload released content.  We can get into these later.
\end{gram}

\section{Books}

\begin{gram}[Create a book]
You can create new books and as many as you want.  I usually create one for textbook, one for recitations, one for assignments, one for additional materials (misc).  Many courses create a "Slides" book.  

To create a book go to "links" right next to search bar and then press
manage books.  You will give your book a unique label (e.g,
"textbook", "recitations", "misc"...) and a unique"rank" which is the
order it will be shown but is otherwise immaterial.  You will see a
book is actually a "booklet" by default.  A booklet consists of a
sequence of chapter.  If you want "parts" and chapters, then uncheck
this.  For example, in my algorithms course, the main textbook has
parts but the rest don't.  A book can contain PDFs or XMLs (generated
by my compiler).
\end{gram}


\begin{gram}
To edit the book, create and upload chapters, go to "Manage Books" and select your book.  Now you can create a chapter and edit existing chapters.
\end{gram}

 

\section{User Accounts} 

\begin{gram}[Creation]
To create user accounts, you can upload a roster in CSV format, which will email them all.  I usually wait to do that until the first day of the class or right before it but you can create accounts for your TAs a bit ahead of time so that they can start playing with things (TAs and undergrads seem to pick things up super quickly).  When you create accounts, the students receive email notifications and instructions to log on.
\end{gram}

